%% LyX 2.1.2.1 created this file.  For more info, see http://www.lyx.org/.
%% Do not edit unless you really know what you are doing.
\documentclass[english]{scrartcl}
\usepackage[T1]{fontenc}
\usepackage[latin9]{inputenc}
\usepackage{geometry}
\geometry{verbose,tmargin=1cm,bmargin=1cm,lmargin=1cm,rmargin=1cm}
\usepackage{units}
\usepackage{textcomp}
\usepackage{amsmath}
\usepackage{amssymb}
\usepackage{esint}
\PassOptionsToPackage{normalem}{ulem}
\usepackage{ulem}
\usepackage{babel}
\begin{document}

\title{Deep learning review}

\maketitle
\tableofcontents{}


\section{Introduction}

The challenges of Artificial Intelligence (AI) is to seek that a computer
could exhibit signs of what we dub as intelligence. The field of \emph{perceptual}
AI is itself based on the hypothesis that it could be possible to
teach a machine to experience and respond similarly to perceptual
events (such as musical events) as a human expert would \cite{humphrey2013feature}.
Even though this stimulating field raised interest for several decades,
as the notion of \emph{intelligence }seems daunting to define, researchers
focused on the notions of \emph{machine learning }where a computer
could learn to perform a specific set of tasks. Despite ongoing efforts
in all information retrieval and data mining fields, most of the consistently
evaluated tasks seem to be converging towards pervasive performance
ceilings across many open questions in each respective discipline,
which stands under expected accuracies \cite{casey2008content}. Furthermore,
it has been repeatedly shown that the performance of most state-of-art
algorithms deteriorate significantly when applied to more realistic
and larger datasets \cite{ellis2012large}.

\textbf{=> Common techniques build on trying to decompose into sub-problems}

\textbf{=> Principle of increasingly complex abstractions}

Most current machine learning techniques rely on shallow architectures,
where (notwithstanding eventual data pre-processing), only a single
layer of nonlinear transformation is applied to a set of features
that are in turn fed to a discriminative pattern recognition algorithm.
However, research in human congnition \cite{hupe1998cortical} suggest
that deep architectures might be implemented by human perception mechanisms
for extracting complex structure out of raw information. Hence this
depth of processing would seem like a natural choice for building
internal representation of increasingly higher-level concepts. These
types of clearly layered hierarchical structures seems logical by
simply thinking about our abilities to decipher and transform information
spanning from the raw input level up to the linguistic and even paradigmatic
analysis level.

\textbf{=> Limits of features}

Several authors pointed out the sub-optimality and limitations of
hand-designing features. Furthermore, current learning systems are
also bound to shallow architectures which are inherently limited.

Deep learning is based on the hypothesis that each layer of processing
tries to remove a different kind of variance by exploiting the statistical
regularities of the current level of abstraction. Deep architectures
can be seen as a set of increasingly higher-level abstractions that
decompose a complex problems into a hierarchical array of simpler
ones. Many researchers tried to train deep architectures for decades
without any success \cite{utgoff2002many}. However, interpreting
each layer as a singular problem of finding optimal regularities to
move from one level of abstraction to a higher-level one allows to
decompose the deep learning in itself. This leads to one of the core
component in the success of current deep learning approaches being
the use of a \emph{greedy layer-wise unsupervised training} which
was introduced in the work of Hinton et al. \cite{hinton2006fast}.
This allows to train each layer separately, alleviating the problems
of training deep architecture through a method with a time complexity
only linear with the depth of the network.

This hierarchy of non-linear processing layers provides a workflow
in which the outputs of each layer provide a certain level of abstraction
which is in turn fed to its higher layer as an input to seek an even
higher-level abstraction for extracting structures by exploiting statistical
regularities in this input.

\textbf{=> Deep learning can be seen as transforming a representation
into another (+ distributed representation)}

Manually crafted features are inherently limited, if not only by the
knowledge of their designer, but also in their restricted ability
to generalize and adapt to different problems and datasets. Furthermore,
shallow architectures are fundamentally limited by the very nature
of their structure which cannot leverage different granularity of
knowledge representation.

Recent works \cite{le2013building} even try to adapt these ideas
to large-scale datasets and massive amounts of data processing.


\subsection{Advantages of deep architectures}

Most machine learning approaches have gradually converged towards
a deep architecture. Indeed, if each transformation of the data is
thought of as a layer of processing, then we can see that most research
is based on the gradual expansion of systems with additionnal steps.
By reducing these operations to the three major types of affine (linear)
combination, non-linear mapping and grouping (or pooling), we can
see that all methods can be cast into the framework of deep architectures.
Deep learning can be seen as a generalization of current research
if common processing structures are seen as transformative layer (filtering,
non-linearity and grouping).

The number of parallel units required in a transformation function
defines its breadth \textbf{{[}xREFx{]}}. There is a clear trade-off
between the breadth of a structure and its depth. 

Deep architectures also exploit the characteristic of \emph{emergence
}which appears where the combination of simpler elements creates a
higher-level composed object that is superior to the sum of its parts.
This plays a fundamental role in the representational power of deep
structures as the ability to perform multiple non-linear combination
of small components lead to more versatile expressivity. In this endeavor,
the use of non-linearity at each level is the key aspect of the expressive
power of deep architectures. Indeed, a combination of linear operations
of any depth can always be reduced to a linear transformation, whereas
non-linear operation provide increasingly complex interaction as a
function of depth. This property of deep architectures produce an
exponential growth of different processing \emph{pathways }which provides
variety of paths reusing their various computational parts \cite{bengio2009learning}.


\subsection{Feature learning}

The consensus growing through the past decade of machine learning
has shown the fundamental importance of powerful and expressive features
representing the data. In the common two-stage supervised learning
paradigm of first extracting features from the raw input and then
providing a semantic interpretation, it would even seem that a naive
classifier is sufficient as long as the features are optimal for the
task at hand. This is driven by the simple observation that irrelevant
variations or noise embedded in a feature must be alleviated through
the pattern recognizer instead. Hence, the more accurate and robust
is a feature and the simpler the pattern matching approach needs to
be (and conversely).

The expressivity of a representation is highly tied to its ability
to its understand and disentangle the underlying factors of variation.
Hence, a good representation is targeted to simultaneously remove
irrelevant variance and emphasize the critical explanatory variables
hidden in the low-level data \cite{bengio2013representation}. This
in turn imply that a good representation will provide a high similarity
between two elements with similar abstract features, even though their
raw values are distant.

Traditionnaly, features are designed by leveraging an intellectual
expectation on what characteristics should be extracted from an input
and trying to reify this concept. This expectation is usually based
on what explicit concept we seek to determine and discriminate. Therefore,
hand-crafted features are bound to the representational expressivity
that we can theorize and bound to our capacity to encode it. Furthermore,
the comparative quality of these features can only be evaluated through
their use in a supervised pattern recognition framework and not directly.
Hence, the expressivity of these features is measured by their ability
to provide an accurate prediction of data, in which case they are
said to be robust.

However, this design scheme seems to be limited by our own experience
and expectations on the shape of the optimization. However, this limits
the potential exploration of the feature space and if our own priors
on the data turns out to be only speculative, we might even be unwillingly
constraining the final search space to only sub-optimal regions. Furthermore,
features are usually repurposed from one problem to another for which
they have clearly not be designed for, which might even worsen their
sub-optimality.

However, feature extraction is typically constructed as an ordered
combination of sets of (linear and non-linear) operations. Hence,
instead of relying on manual search and optimization, this process
could be reified through an automatic procedure of finding the optimal
layering of operations. Therefore, feature design itself can be seen
as a learning problem with its own search space, with the goal of
finding the optimal transformation of input data that leads to the
most expressive representation. Deep learning provides both a flexible
and elegant solution to this problem by generalizing the feature and
semantic optimization steps into the same framework.

\textbf{+ Idea of }\textbf{\emph{distributed representation ?}}

\textbf{+ Pitfalls of ``local template matching ?''}


\subsection{Manifold and complexity}

The feature learning paradigm is also driven by a strong underlying
hypothesis that within the full space of possible datas ($\mathbb{R}^{n}$
for $n-$dimensional vectors), the \emph{real} data lives on a complex
and highly non-linear manifold of lower dimensionality which occupies
only a small portion of this full space. In this case, the input space
is said to be \emph{over-complete }with respect to the data. Hence,
being able to accurately model this hypothetical manifold embedded
within the input space would allow to disentangle the uninformative
variance and provide the optimal data representation.

Under this assumption, the low-order, shallow architectures would
be ill-equiped to accurately represent the complexity of this manifold
accurately (as the complexity of data distribution would exceeds that
of the model). Indeed, in order to cope with the large number of non-linear
boundaries of this manifold, simple model would need to compensate
with a very wide number of piece-wise approximations. In this sense,
deep architectures can leverage the non-linear combinations to provide
an exponentially increasing model complexity.

These ideas of targeting directly a model of this manifold are being
increasingly studied, notably in the field of \emph{representation
learning} \cite{bengio2013representation}. This is also one of the
major properties of the generalization properties of deep architectures.
The learning procedures are usually targeted at giving low reconstruction
error on the data points coming from the same data-generating distribution
as the training set, while having high reconstruction error on random
sample of the full input space. Hence, deep architectures indirectly
model this non-linear manifold. This idea has also been directly targeted
through the notion of space embedding \cite{weston2012deep} or with
special types of penalties \cite{rifai2011contractive}.


\subsection{Application to audio}

Application of deep learning to audio have seen a flourishing literature
in the recent years \cite{yu2011deep}. As put forward by \cite{humphrey2013feature},
the short-time nature of current audio analysis algorithm is inherently
unable to encode musically meaningful structure at the track level.

Music is deep : \uline{``is that music is composed: pitch and
loudness combine over time to form chords, melodies and rhythms, which
in turn are built into motives, phrases, sections and, eventually,
construct entire pieces. This is the primary reason shallow architectures
are ill-suited to represent high-level musical concepts and structure.
A melody does not live at the same level of abstraction as a series
of notes, but is instead a higher, emergent quality of those simpler
musical objects.''}


\subsubsection{Time scales problem}

Most music processing architectures are based on a short-time analysis
paradigm. However, given the nature and construction of music, this
approach is ill-suited to capture any form of syntagmatic and even
paradigmatic higher-level knowledge. Music usually unfolds over both
longer time-scales and that temporal information is usually fundamental
to the perception of music. Hence, the bag-of-feature approach is
inherently limited to descriptive static aspect frozen in time and
the information transcribed is limited to the time scale of the analysis
itself.

Furthermore, different temporal information coexist at various time
scales which raise the question of temporal granularity. Even though,
it appears quite complex to incorporate multiple and longer time scales,
it is doubtful that any short-time analysis can encode any musically
meaningful information.


\section{Architectures}

\textbf{Globally need to talk about cost functions + gradient}

\textbf{CF BENGIO + Representation learning article}

\uline{To implement this, one would start by manually deriving
the expressions for the gradient of the loss with respect to the parameters:
in this case} $\nicefrac{\partial\ell}{\partial\Theta_{i}}$


\subsection{Logistic regression}

The logistic regression is the simplest form of probabilistic classifying
network composed of a single layer. The underlying idea is that each
class in a supervised problem can be represented by an hyperplane
separating the input space. Hence, by projecting an input vector onto
each hyperplane, we can obtain the distance of this input to the hyperplane,
which reflects its (inverse) probability to belong to the corresponding
class. Formally, we wish to learn the parameters of the hyperplanes,
defined by a weight matrix $W$ and a bias $b$. The probability that
the input $\mathbf{x}$ belongs to a particular class $c$ can be
defined through the \emph{softmax }operator
\begin{eqnarray*}
P\left(Y=c\mid\mathbf{x},\mathbf{W},b\right) & = & softmax_{c}\left(\mathbf{W}\mathbf{x}+b\right)\\
 & = & \frac{e^{\mathbf{W}_{c}\mathbf{x}+b_{c}}}{\sum_{i}e^{\mathbf{W}_{i}\mathbf{x}+b_{i}}}
\end{eqnarray*}


Then we can select the class which imply the highest probability as
the prediction $c_{pred}$ such that 

\[
c_{pred}=\underset{c}{argmax}\left[P\left(Y=c\mid\mathbf{x},W_{c},b_{c}\right)\right]
\]


In order to obtain the best classification accuracy, the parameters
$\theta=\left\{ \mathbf{W},\mathbf{b}\right\} $ of the hyperplanes
are learned through the minimization of the loss function. In this
case, we seek to reduce the number of misclassified example. Thus,
this turns out to be equivalent to maximizing the log-likelihood of
the dataset $\mathcal{D}$ under the parameters $\theta$ defined
by\emph{
\begin{eqnarray*}
\mathcal{L}\left(\theta=\left\{ W,b\right\} ,\mathcal{D}\right) & = & \sum_{i=1}^{\left|\mathcal{D}\right|}log\left(P\left(Y=y^{(i)}\mid\mathbf{x}^{(i)},W,b\right)\right)
\end{eqnarray*}
}

Finally the complete cost function (over a set of $n$ training examples)
used for multi-class logistic regression is defined through the softmax
operator

\emph{
\[
J(\theta)=-\frac{1}{n}\left[\sum_{i=1}^{n}\sum_{j=1}^{k}1\left\{ y^{(i)}=j\right\} \log\frac{e^{\theta_{j}^{T}x^{(i)}}}{\sum_{l=1}^{k}e^{\theta_{l}^{T}x^{(i)}}}\right]
\]
}

where $1\{\cdot\}$ is the indicator function. This leads to the gradient
used for weights update

\emph{
\[
\nabla_{\theta_{j}}J(\theta)=-\frac{1}{n}\sum_{i=1}^{n}\left[x^{(i)}\left(1\{y^{(i)}=j\}-p(y^{(i)}=j|x^{(i)};\theta)\right)\right]
\]
}

$\nabla_{\theta_{j}}J(\theta)$ defines the vector of derivatives
of the cost with respect to the parameters of the model (the i-th
element is the partial derivative of $J(\theta)$ with respect to
the i-th parameter).


\paragraph{Properties}

One interesting property of the softmax classifier is that subtracting
a particular value $\alpha$ from every parameter $\theta_{i}$ does
not affect the prediction, which means that $\mathcal{J}\left(\theta\right)=\mathcal{J}\left(\theta-\alpha\right)$.
Hence, if the cost function $J(\theta)$ is minimized by a set of
parameters $\theta=\left\{ \theta_{1},\ldots,\theta_{n}\right\} $,
then it is also minimized by $\theta^{'}=\left\{ \theta_{1}-\psi,\ldots,\theta_{n}-\psi\right\} $
for any value of $\alpha$, because the parameters of the softmax
regression are redundant. Even though this shows that the minimizer
of $J(\theta)$ is not unique, this property comes in handy to avoid
a potential overflow in the parameters values.


\subsection{Neural networks}


\subsubsection{Artificial neuron}

The computational model of a neuron trace back to the seminal work
of McCulloch and Pitts in 1943 . An artificial neuron is a computational
unit that combines the weighted sum of its input and activates if
this sum is over a threshold. Formally, a neuron is parametrized by
a weight vector $\mathbf{W}\in\mathbb{R}^{n}$ and a bias $b\in\mathbb{R}$.
For a given input vector $\mathbf{x}\in\mathbb{R}^{n}$, the neuron
outputs

\[
{\textstyle h(\mathbf{x})=\phi(\mathbf{W}^{T}\mathbf{x}+b)=\phi(\sum_{i=1}^{n}W_{i}x_{i}+b)}
\]


where $\phi:\mathbb{R}\mapsto\mathbb{R}$ is called the \emph{activation
function. }Hence, we can see that a neuron is a combination of an
affine transform and a non-linearity that decides whether the neuron
activates.


\subsubsection{Various computation units}


\paragraph{Sigmoid}

The most common activation function used in the literature (tracing
back to the original works on neural networks) is the sigmoid function:

\[
\phi\left(z\right)=\frac{1}{1+e^{-z}}
\]


In this case, a single neuron is defined similarly to the mapping
performed by logistic regression.


\paragraph{Hyperbolic tangent}

Another usual choice for the activation function $\phi$ is the hyperbolic
tangent (noted \emph{tanh}), defined as

\[
\phi(z)=tanh\left(z\right)=\frac{e^{z}-e^{-z}}{e^{z}+e^{-z}}
\]



\paragraph{Gaussian}

Gaussian activation functions have been used in the form of \emph{radial
basis functions }(RBFs) in the so-called RBF networks and is defined
as
\[
\phi\left(\mathbf{x}\right)=exp\left(-\frac{\left\Vert \mathbf{x}-\mu\right\Vert ^{2}}{2\sigma^{2}}\right)
\]


These functions are more successful in the case where neural networks
are used as function approximators.


\paragraph{Rectified linear}




\paragraph{Others}

Several other types of computational units have been proposed such
as
\begin{itemize}
\item Multiquadratics and inverse multiquadratics
\item Binary and bipolar step (Heaviside) function
\item Ramp and identity function
\end{itemize}

\subsubsection{Multi-layer neural networks}

It seems clear than from the definition of a single artificial neuron,
we could construct a whole network based on this unit. In this network
the output of a neuron could ``feed'' the input of another one.
Hence the multi-layer neural network (often called \emph{multi-layer
perceptron }(MLP)) are constructed by setting layers of neuron one
over the other. Each layer is densely connected to the next layer,
with the output of one neuron connected to the inputs of all neurons
of the next layer. If we feed an input vector $\mathbf{x}$ to the
network, the neurons with a sufficiently strong weighted input signal
will activate and emit an activation. The activations of this layer
will in turn become the input of the next layer which will perform
the same computations. Hence, given the activation vector $\mathbf{a}^{(l)}$
of a given layer $l$, we can compute the activation of the next layer
$\mathbf{a}^{(l+1)}$ as\emph{ 
\[
a^{(l+1)}=\phi(\mathbf{W}^{(l)}\mathbf{a}^{(l)}+b^{(l)})
\]
}

by setting $\mathbf{a}^{(0)}=\mathbf{x}$ to be the input. This step
is called the \emph{forward propagation}, as the activations are propagated
from one layer to the next. As the network is considered to be densely
connected, we can organize its parameters in matrices and use matrix-vector
operations to take advantage of fast linear algebra routines to perform
quick calculations.


\subsubsection{Learning algorithm}

If the goal of the network is to output a known (desired) vector of
values $\mathbf{y}$, we can compute the amount of error in the approximation
made by the network. This will subsequently allow the network to learn
from its own mistakes (by comparing the output of its last layer $h\left(\mathbf{x}\right)$
to the desired one $\mathbf{y}$). This function, called the \emph{cost
function} can be defined for a single input vector $\mathbf{x}$ as

\[
\mathcal{J}(W,b;\mathbf{x},\mathbf{y})=\left\Vert h(\mathbf{x})-\mathbf{y}\right\Vert ^{2}
\]
This squared-error cost function simply evaluate the Euclidean distance
between the output produced by the network and the desired output
$\mathbf{y}$. Given the entire set of $n$ training examples, we
can define the global cost function as

\[
\mathcal{J}\left(W,b\right)=\left[\frac{1}{n}\sum_{i=1}^{n}\left(\left\Vert h(\mathbf{x}^{(i)})-\mathbf{y}^{(i)}\right\Vert ^{2}\right)\right]+\lambda\sum_{i,j,l}\left(W_{ji}^{(l)}\right)^{2}
\]


The first term simply defines the (squared) \emph{prediction error
}of the network. The second term is used to bound the magnitude of
the weights which is intended to prevent \emph{overfitting} (as unbounded
weights can also increase linearly in the same direction over learning).
The term (often called the \emph{weight decay penalty}) acts as a
regularization and its impact is controled by the \emph{weight parameter
}$\lambda$.

In order to learn in this network, we need to first initialize the
parameters randomly and then evaluate the error of the network. Then,
by looking at the gradient of the error with respect to each parameter
(how much will the error increase or decrease depending on how we
change this parameter), we can find the best direction to take.

It is important to note that the parameters \emph{must be} initialized
with small random values instead of all zeros. Indeed, if all the
parameters are set to the same values, then all neurons would provide
the same output and, thus, the error gradients would be similar for
all units, ending in an array of units learning the exact same function.
Therefore, this random initialization allows to enforce a \emph{symmetry
breaking}.

The derivative of a complete network can be a complex task. Fortunately,
the backpropagation algorithm was developped to simplify the updates
of a function network, by seeing that the gradients are usually functions
of the gradients of deeper layers. The complete algorithm is as follows
\begin{enumerate}
\item Perform a forward propagation in the network in order to obtain the
activations $a^{(l)}$ for each layer up to the last (output) layer
$a^{(n_{l})}$
\item In the case of a squared error, the output layer partial derivative
is given by 
\begin{align*}
\delta^{(n_{l})}=-2(y-a^{(n_{l})})\bullet\phi'(z^{(n_{l})})
\end{align*}

\item For all previous layers, we can compute the partial derivatives 
\begin{align*}
\delta^{(l)}=\left((W^{(l)})^{T}\delta^{(l+1)}\right)\bullet\phi'(z^{(l)})
\end{align*}

\item The final derivatives with respect to different parameters is given
by 
\begin{align*}
\nabla_{W^{(l)}}\mathcal{J}(W,b;x,y) & =\delta^{(l+1)}(a^{(l)})^{T}\\
\nabla_{b^{(l)}}\mathcal{J}(W,b;x,y) & =\delta^{(l+1)}
\end{align*}

\end{enumerate}

\subsubsection{Gradient descent}

In order to update the parameter, we need to take a step in the direction
opposite to the gradient. Indeed, if the gradient is decreasing, the
error is decreasing in the direction of increasing the parameter value.
Therefore, we can simply use the value of this gradient 

\[
\theta_{i}^{(l)}=\theta_{i}^{(l)}-\epsilon\left(\frac{\partial\mathcal{J}(W,b;x,y)}{\partial\theta_{i}^{(l)}}\right)
\]


where $\epsilon$ is called the \emph{learning rate }which controls
the size of the step we will take at each iteration. Although it would
seem natural to set this parameter to the maximal value, this increases
the risk of overshooting (where the update makes steps so wide that
it can ``jump over'' the optimal values). Oppositely, setting a
low learning rate will cause the algorithm to be very slow to converge.

As the learning rate appears to be the most sensitive parameters of
the learning, several variations have been devised such as \emph{adadelta}
\cite{duchi2011adaptive}, which automatically tunes this parameter
in order to quickly converge towards the local optimum. Other more
sophisticated methods such as L-BFGS rely on quasi-Newton's method
by trying to approximate the Hessian matrix to provide faster convergence
\cite{liu1989limited}.


\subsection{Difficulty of training deep architectures}

The major obstacle in training and learning deep networks is the omnipresence
of local optima in the objective function of the deep networks \cite{yu2011deep}.
When trying to apply backpropagation to optimize the wide array of
network parameters, these are usually initialized with randomly distributed
points. Hence, depending on the position of this randomly chosen starting
point, the subsequent local gradient descent can easily get trapped
in a local optima and this pervasive problem increases significantly
with the network depth, as it increases the number of parameters and
number of local optima \cite{bengio2007greedy}.
\begin{itemize}
\item \emph{Lack of availability of labeled data required for supervised
training.}
\item \emph{On the pervasive presence of local optima}.
\item \emph{Diffusion of the error gradient}. When performing the back-propogation
algorithm, the error gradients are computed at the final layer and
then propagated backwards. However, at each step towards a previous
layer, these are multiplied by the derivative of the current layer
(typically constrained by regularization to remain small). Therefore,
the gradients and their impact on the weights will quickly diminish
in magnitude as the depth of the network increases. Therefore, the
impact of the derivative of the overall cost becomes less and less
significant at each earlier layers, which greatly reduces the ability
of these layers to learn.
\end{itemize}

\subsection{Unsupervised (self-taught) learning}

It is well-known that in machine learning problems, best performance
can be achieved simply by feeding more data to the algorithms, sometimes
even overshadowing the strength of the algorithms themselves. To achieve
this goal, the unsupervised feature learning (sometimes termed \emph{self-taught
learning)} framework holds the promise that algorithms could learn
from any unlabeled data. As the only constraint is that this data
should be of the same \emph{nature }as the studied problem, massive
amounts of data can be easily obtained to perform learning. However,
self-taught learning setting does not assume that the unlabeled data
is drawn from the same data-generating distribution as the labeled
data. The only requirement is that dimensions are of the same nature.
This would allow to learn the underlying structure of the data from
a wide array of unlabeled examples and then fine-tuning the learning
from a smaller amount of labeled training data to target a specific
supervised task. Fine-tuning significantly improves the classifying
performance by exploiting the labeled training set to more closely
fit to the statistical regularities of a specific problem.

Based on a learned network of abstractions, feeding an input vector
to it provides a vector of activations in the last layer. These activations
are supposedly a higher-level and more efficient representation of
the input. Then, we can either just \emph{replace} the original vector
with the activation vector or \emph{concatenate} the two feature vectors
(this can also be seen as a network where both the activation and
the input directly feed directly the classifying layer). However,
it has been shown that this concatenation representation provides
only marginal changes to the replacement operation \textbf{{[}xREFx{]}}


\subsection{Greedy layer-wise training}

\textbf{\uline{MERGE WITH UPWARDS}}\uline{ }\textbf{\uline{OR
MAYBE TRASH}}\uline{ }\textbf{\uline{(BIS REPETITA) }}\uline{One
method that has seen some success is the greedy layer-wise training
method. Training can either be supervised (with classification error
on each step), but more frequently it is unsupervised. This method
trains the parameters of each layer individually while freezing parameters
for the remainder of the model. The weights from training the layers
individually are then used to initialize the weights in the final/overall
deep network, and only then is the entire architecture \textquotedbl{}fine-tuned\textquotedbl{}
(i.e., trained together to optimize the labeled training set error).}
\begin{itemize}
\item \emph{Bypassing the lack of labeled data}. As the self-taught and
greedy layer-wise learning approaches are able to exploit massive
amounts of unlabeled data (given the only requirement that these data
are of same nature, i.e. pertain to the same underlying data-generating
distribution), these methods bypass the need for large collections
of labeled data. Furthermore, using unlabeled provides strongly better
initial values for the weights in all pretraining layers (except the
final classification layer that requires labeled data for training).
Hence, these algorithms can uncover more robust patterns by thriving
on massively more amounts of data than supervised approaches.
\item \emph{Better local optima}. By starting at a more advantageous region
of the parameter space than with random initialization, fine-tuned
networks starting from this more optimal location can lead to better
local optima. Intuitively, gradient descent from a pre-trained location
embeds a significant amount of \textquotedbl{}prior\textquotedbl{}
knowledge extracted from the underlying structure of unlabeled data.
\item \emph{Providing a logical decomposition of a task into a set of sub-problems
at different levels of abstraction}
\end{itemize}
It has been shown that this unsupervised pre-training approach provides
a strong data-driven prior \cite{hinton2006fast}, which can be seen
as a form of regularization \cite{erhan2009difficulty}. Indeed, by
exploiting the structure of the nature of the data itself (independently
of the eventual supervised task at hand), the feature learning layers
are initialized in a more advantageous region of the parameter space,
which provides better local optima and requires less labeled data
for converging to an adequate network.


\subsection{Reconstruction goal}

Even though the general idea of layer-wise training appears as an
ideal way to handle deep architectures, its pragmatic learning goal
(or objective function) remains to be defined. We know that we hope
to learn at each layer a higher-level abstraction based on statistical
regularities of the previous lower levels. So the guiding principle
for learning these intermediate representations should be to capture
the most relevant information of the underlying distribution of lower-level
concepts. In order to define this in a pragmatic learning function,
the key lies in the concept of \emph{reconstruction}. A concept from
a higher layer should be constructed out of a set of concepts from
the lower layers. So if we are able to decompose a specific abstraction
in a limited set of pieces and then can reconstruct it back from this
set, it means that we have discovered its most salient components.
Formally, the objective function is to learn an encoding function
$e$ that decompose the input and a decoding function $d$ that reconstruct
this input from its code. These functions should minimize the error
between an input $x$ and its version $\tilde{x}$ reconstructed from
these function.
\begin{eqnarray*}
Y & = & e_{\theta_{e}}\left(X\right)\\
\tilde{X} & = & d_{\theta_{d}}\left(Y\right)
\end{eqnarray*}


As can already be seen, an amount of regularization will be necessary
to prevent the algorithm from learning the identity function. Furthermore,
we would also like to ensure that the representation learned by the
system remains robust to small variabilities in the input (notably
with respect to the \emph{manifold }hypothesis).


\section{Single-layer modules}

Although targeted at learning deep architectures, there is a broad
division in research depending on the interpretation given to the
connexionnist architecture. First, the probabilistic view lead to
models driven by probabilistic graphical models, interpreting the
hidden units as latent random variables. Second, following the research
on neural networks lead to model constructed through computation graphs,
where hidden units are considered as computational nodes. These two
views are not entirely dichotomic as their similarities seem to outweigh
their differences. Indeed, it has been shown that these two views
are in fact almost equivalent under certain assumptions \cite{vincent2011connection}.


\subsection{Auto-encoders}

The Auto-Encoder (AE) was first introduced \cite{rumelhart1988learning}\textbf{
}as a dimensionnality reduction technique. In its original formulation,
an AE is composed of an encoder and a decoder, where the output of
the encoder provides a reduced (compressed) representation of the
input and the decoder allows to reconstruct the orginal input from
this encoded representation. Hence, both the encoding and decoding
part are tuned through the minimization of a reconstruction error
function, which finds a non-linear dimensionality reduction and representation
fit to a certain dataset (as the encoder has a lower number of units
than the input data). 

We can see that the framework defined by AEs fit the overarching goal
of unsupervised and self-taught learning, as it tries to exploit statistical
correlations of the data structure to find a non-linear representation
aimed at decomposing and then reconstructing the input. However, when
trying to learn increasingly higher-level abstractions, it seems more
logical to have an increasing explanatory power through higher dimensionnalities.
Hence, as opposed to their historical dimensionality-reduction objectives,
current auto-encoders are defined as \emph{over-complete} (with an
encoder layer of higher dimensionality than the input). This definition
is aimed at extracting a set of features larger than the input.

It has been shown that when a linear activation function is used in
the encoding layer and that it forms a bottleneck by having a number
of units inferior to the input dimensionality, the learnt parameters
of the encoder are a subspace of the input space principal components
\cite{baldi1989neural}. \textbf{\uline{MAYBE TRASH/MERGE}}\uline{This
is similar to the way the projection on principal components would
capture the main factors of variation in the data. Indeed, if there
is one linear hidden layer (the code) and the mean squared error criterion
is used to train the network, then the k hidden units learn to project
the input in the span of the first k principal components of the data}\textbf{\uline{
MAYBE TRASH/MERGE}}\uline{. }However, the use of non-linear activation
functions in the encoder provides a more expressive framework and
lead to more useful feature-detectors than what can be obtained with
a simple PCA as it provides a non-linear transformation of the input
data.


\subsubsection{Basic auto-encoder}

The autoencoder aims at learning both an encoding function $e$ and
a decoding function $d$ such that ${\textstyle d\left(e\left(\mathbf{x}\right)\right)=\tilde{\mathbf{x}}\approx\mathbf{x}}$.
Therefore, the AE is intended to learn a function approximating the
identity function, by being able to reconstruct an ${\textstyle \tilde{\mathbf{x}}}$
similar to the input $\mathbf{{\textstyle x}}$ via a hidden representation.
In the case of entirely random input data (for instance a set of IID
Gaussian noise), this task would not only be hard but also quite meaningless.
However, based on the assumption that there might exist an underlying
hidden structure in the data (where part of the input features are
consistently correlated), then this approach might be able to uncover
and exploit these statistical regularities. The encoding function
$e:\mathbb{R}^{d_{x}}\rightarrow\mathbb{R}^{d_{h}}$ maps an input
$\mathbf{x}\in\mathbb{R}^{d_{x}}$ to an hidden representation $\mathbf{h}_{\mathbf{x}}\in\mathbb{R}^{d_{h}}$
by producing a deterministic mapping 
\[
\mathbf{h}_{\mathbf{x}}=e\left(\mathbf{x}\right)=s_{e}\left(\mathbf{W_{e}}\mathbf{x}+\mathbf{b}_{e}\right)
\]


where $s_{e}$ is a nonlinear activation function (usually the \emph{sigmoid
}function), $\mathbf{W}_{e}$ is a $d_{h}\times d_{x}$ weight matrix
, and $\mathbf{b}_{e}\in\mathbb{R}^{d_{h}}$ is a bias vector.

The decoding function $d:\mathbb{R}^{d_{h}}\rightarrow\mathbb{R}^{d_{x}}$
then maps back this encoded representation $\mathbf{h_{x}}$ into
a reconstruction $\mathbf{y}$ of the same dimensionnality as $\mathbf{x}$
\[
\mathbf{y}=d\left(\mathbf{h_{x}}\right)=s_{d}\left(\mathbf{W}_{d}\mathbf{h_{x}}+\mathbf{b}_{d}\right)
\]


where $s_{d}$ is the activation function of the decoder. Usually
the weight matrix of the decoding layer $\mathbf{W}_{d}$ is tied
to be the transpose of the encoder weight matrix $\mathbf{W}_{d}=\mathbf{W}_{e}^{T}$,
in which case the AE is said to have \emph{tied weights}.

Hence, training an auto-encoder can be summarized as finding the optimal
set of parameters $\theta=\left\{ \mathbf{W}_{e},\mathbf{W}_{d},\mathbf{b}_{e},\mathbf{b}_{d}\right\} $
(or $\theta=\left\{ \mathbf{W},\mathbf{b}\right\} $ in the case of
tied weights) in order to minimize the reconstruction error on a dataset
of training examples $\mathcal{D}_{n}$ 
\[
\mathcal{J}_{AE}\left(\theta\right)=\sum_{\mathbf{x}\in\mathcal{D}_{n}}\mathcal{L}\left(\mathbf{x},d\left(e\left(\mathbf{x}\right)\right)\right)
\]


Usual choices for the reconstruction error function $\mathcal{L}$
are either the squared error $\mathcal{L}(x,y)=\left\Vert x-y\right\Vert ^{2}$
(often used for linear reconstruction) or the cross-entropy loss of
the reconstruction $\mathcal{L}(x,y)=-\sum_{i=1}^{d_{x}}x_{i}log\left(y_{i}\right)+\left(1-x_{i}\right)log\left(1-y_{i}\right)$
(if the input is interpreted as vectors of probabilities and a sigmoid
activation function is used). 

As can be seen from the definition of the objective functions, by
solely minimizing the reconstruction error, nothing prevents an auto-encoder
with an input of $n$ dimensions and an encoding of the same (or higher)
dimensionnality to simply learn the identity function. In this case,
the AE would merely be mapping an input to a copy of itself. Surprisingly,
it has been shown that non-linear autoencoders in this \emph{over-complete}
setting (with a hidden dimensionality strongly superior to that of
the input) trained with stochastic gradient descent, could still provide
useful representations, even without any additional constraints \cite{bengio2007greedy}.
\textbf{NON-LINEARITY ALREADY ACTS AS A REGULARIZER + }\uline{A
simple explanation is that stochastic gradient descent with early
stopping is similar to an L2 regularization of the parameters. To
achieve perfect reconstruction of continuous inputs, an auto-encoder
with non-linear hidden units needs very small weights in the first
(encoding) layer, to bring the non-linearity of the hidden units into
their linear regime, and very large weights in the second (decoding)
layer. It means that the }\emph{\uline{representation is exploiting
statistical regularities present in the training set}}\uline{,
rather than merely learning to replicate the input.}

Nonetheless, several penalties + \textbf{bla bla bla}


\subsubsection{Regularized auto-encoders}

\emph{Weight-decay} is the simplest regularization technique targeted
at preventing overfitting by favoring smaller weights in the learning.
The idea is to add a penalty term on the magnitude of weights to the
overall cost function 
\[
\mathcal{J}_{AE}\left(\theta\right)=\left(\sum_{\mathbf{x}\in\mathcal{D}_{n}}\mathcal{L}\left(x,d\left(e\left(\mathbf{x}\right)\right)\right)\right)+\lambda\sum_{ij}W_{ij}^{2}
\]


where $\lambda$ is a parameter controlling the impact of the regularization
on the global cost.


\subsubsection{Sparse auto-encoders}

One solution to avoid the degeneracy and prevent the AE for learning
the identity functio is to add a \emph{sparsity} constraint to the
cost function by forcing many of the hidden units to be zero or near-zero
\cite{ranzato2007sparse,lee2008sparse}. Hence, by imposing this sparsity
constraint, we can uncover interesting structures even when the number
of hidden units is to the size of the input. The goal of finding this
efficient representation would be to ensure that most hidden units
are inactive for a large portion of the dataset (ie. features learned
are \emph{specific}). For each hidden unit $i$, we can compute its
average actiavtion (across the set of $n$ training examples) 
\begin{align*}
\hat{\rho}_{i}=\frac{1}{n}\sum_{j=1}^{n}\left[a_{i}(\mathbf{x}_{j})\right]
\end{align*}
For the AE to be sparse, we would like the units to only activate
on specific subsets, which is equivalent to enforcing the constraint
for each hidden unit that $\hat{\rho}_{i}=\rho$ where ${\textstyle \rho}$
is called a sparsity target, typically close to zero. This would force
the network to learn being able to reconstruct the input with a very
limited number of units. To achieve this, we can add a penalty term
to the overall optimization objective in order to penalize the mean
activation of each unit ${\textstyle \hat{\rho}_{i}}$ that deviates
significantly from the sought sparsity ${\textstyle \rho}$. This
can be achieved by minimizing the Kullback-Leibler (KL) divergence
between all ${\textstyle \hat{\rho}_{i}}$ and $\rho$, by considering
the average activation and sparsity target as Bernoulli random variables

\[
\sum_{i}{\rm KL}(\rho,\hat{\rho}_{i})=\sum_{i}\rho log\frac{\rho}{\hat{\rho}_{i}}+(1-\rho)log\frac{1-\rho}{1-\hat{\rho}_{i}}
\]


Therefore, the complete cost function to minimize is defined by the
sum of the normal cost function while accounting for this sparsity
constraint defined as

\[
\mathcal{J}_{{\rm sparse}}\left(\theta\right)=\mathcal{J}\left(\theta\right)+\beta\sum_{i}{\rm KL}\left(\rho,\hat{\rho}_{i}\right)
\]


where $\beta$ controls the influence of the sparsity constraint on
the global cost, and the derivative of the constraint is given by
\[
\delta_{sparse}=\beta\left(-\frac{\rho}{\hat{\rho}_{i}}+\frac{1-\rho}{1-\hat{\rho}_{i}}\right)
\]


As we need to know the average activation value ${\textstyle \hat{\rho}_{i}}$
for each hidden unit, this penalty requires to compute a first forward
pass on all the training dataset before computing the derivatives
for each example. Overcomplete AEs with sparsity can be seen as an
attempt to learn a series of traditionnal AE for each of the types
of training data, which might also share some of their hidden structures.


\subsubsection{Denoising auto-encoders}

Another very successful way to regularize AEs have been proposed by
Vincent et al. \cite{vincent2010stacked}, through the concept of
Denoising Auto-Encoders (DAE). Intuitively, the main idea is to corrupt
the input $\mathbf{x}$ in order to produce\textbf{ }a ``noisy''
version $\tilde{\mathbf{x}}$ before passing it to the AE, but then
training the network with the goal to reconstruct the original (clean)
version of $\mathbf{x}$ (producing an overall denoising process).
Therefore, a DAE simultaneously tries to find a robust encoding of
the input, while trying to remove the effect of a stochastic corruption
process applied to its input in order to capture the relevant statistical
dependencies in the input components. Training the autoencoder to
reconstruct a clean input from a corrupted version of itself forces
the hidden layer to uncover robust features but also inherently prevents
it from learning the identity function. Hence, the objective function
for DAE is defined as

\[
\mathcal{J}_{DAE}\left(\theta\right)=\sum_{\mathbf{x}\in\mathcal{D}_{n}}\mathbb{E}_{\tilde{\mathbf{x}}\sim q\left(\mathbf{x}\mid\tilde{\mathbf{x}}\right)}\left[\mathcal{L}\left(\mathbf{x},d\left(e\left(\tilde{\mathbf{x}}\right)\right)\right)\right]
\]


where the corrupted versions $\tilde{\mathbf{x}}$ are obtained by
applying a stochastic corruption process $q\left(\tilde{\mathbf{x}}\mid\mathbf{x}\right)$
to the input examples $\mathbf{x}$. Any type of corruption process
can be considered, typically including additive Gaussian noise and
binary masking (a randomly selected subset of input components are
set to 0). Variable amounts of corruption (variance of the Gaussian
noise or number of dropped components) can be considered to control
the degree of regularization. DAEs may be interpreted from a stochastic,
information theoretic, generative or manifold learning, perspective
\textbf{+ ADD REFERENCES FOR EACH OF THE PERSPECTIVE HERE}

\textbf{+ ADD THE NEW ``INFINITE DENOISE AE'' HERE}


\subsubsection{Contractive auto-encoders}

Recently, another form of regularization called \emph{contractive
}\cite{rifai2011contractive} have been proposed to produce the contractive
auto-encoders (CAE). The main idea behind CAE is to add a penalty
term to the cost function based on the derivative of the hidden features
with respect to the input. Hence, this encourages the learning to
uncover features that have low variations in the local variations
directions found directly in the data. This penalty term can be computed
by taking the Frobenius norm of the Jacobian matrix of the non-linear
encoding (hidden activations) with respect to the input. This penalty
term might yield more robust features by creating contraction in the
space localized around the training examples. This idea can be seen
as a direct attempt to model the existence of a lower-dimensional
non-linear manifold inside the complete input space.

Penalizing the norm of the Jacobian $J_{e}\left(\mathbf{x}\right)$
of the hidden mapping implicitly penalizes its \emph{sensitivity}
to a given input and encourages the robustness of the representation
to slight variations in the input. Formally, given an input $\mathbf{x}\in\mathbb{R}^{d_{x}}$
mapped to a hidden representation $\mathbf{h}\in\mathbb{R}^{d_{h}}$
(through an encoding function $e$), the contractive penalty term
is given by the sum of the partial derivatives of the representation
with respect to the input components
\[
\left\Vert J_{e}\left(\mathbf{x}\right)\right\Vert _{F}^{2}=\sum_{ij}\left(\frac{\partial h_{j}\left(\mathbf{x}\right)}{\partial x_{i}}\right)^{2}
\]


\uline{Penalizing} $\left\Vert J_{e}\left(\mathbf{x}\right)\right\Vert _{F}^{2}$
\uline{encourages the mapping to the feature space to be }\emph{\uline{contractive}}\uline{
in the neighborhood of the training data. The }\emph{\uline{flatness}}\uline{
induced by having low valued first derivatives will imply an }\emph{\uline{invariance}}\uline{
or }\emph{\uline{robustness}}\uline{ of the representation for
small variations of the input.}

The complete cost function of the CAE is given by

\[
\mathcal{J}_{AE}\left(\theta\right)=\sum_{\mathbf{x}\in\mathcal{D}_{n}}\left(\mathcal{L}\left(\mathbf{x},d\left(e\left(\mathbf{x}\right)\right)\right)+\lambda\left\Vert J_{f}\left(\mathbf{x}\right)\right\Vert _{F}^{2}\right)
\]


While DAEs (indirectly) encourages the robustness of the reconstruction
through corruption, CAEs tries to analytically encourage the robustness
of representation itself by penalizing the magnitude of its variations
in the neighborhood of training points. In the case of a sigmoid activation
function, the contractive penalty can be simply calculated with

\[
\left\Vert J_{e}\left(\mathbf{x}\right)\right\Vert _{F}^{2}=\sum_{i=1}^{d_{h}}\left(h_{i}\left(1-h_{i}\right)\right)^{2}\sum_{j=1}^{d_{x}}W_{ij}^{2}
\]


As we can see, in the case of a \emph{linear }encoder (with an identity
activation function), this penalty term is strictly equivalent to
$L^{2}$ weight decay. One problem with the penalty introduced by
CAE is that it might be limited to only \emph{infinitesimal }variations
in the input (because of the use of the first-order derivative). An
extension to all higher-order derivatives has been proposed \cite{rifai2011higher}
leading to the CAE+H model.

\uline{A high dimensional Jacobian contains directional information:
the amount of contraction is generally not the same in all directions.
The Frobenius norm measures the contraction of the mapping }\emph{\uline{locally}}\uline{
at that point. by the ratio of the distances between two points in
their original (input) space and their distance once mapped in the
feature space.}


\subsubsection{Linear decoders}

Because of the use of a sigmoid activation function in the decoding
units of the autoencoders, the inputs are constrained to lie in the
$\left[0,1\right]$ range (as the sigmoid only outputs numbers in
that range). This can turn out to be problematic as most real-life
data is not constrained to the unit range and there might not be a
non-lossy way to scale the input data in this range. An easy way to
alleviate this problem would be to simply remove the sigmoid function
to obtain $a=z$. Formally, this can be achieved by replacing the
sigmoid function in the output nodes by the identity function $\phi(z)=z$
(called in this case a \emph{linear} activation function). The hidden
layer of the AE, however, can still rely on a sigmoid activation function.
An autoencoder with a sigmoid hidden layer and a linear output layer
is called a \emph{linear decoder}. This type of autoencoder can be
trained directly with real-valued inputs without the need to scale
them to a specific range.

The only modification in the linear decoder is to replace the derivation
in the last layer with

\[
\delta_{i}^{(3)}=-(y_{i}-\hat{x}_{i})
\]


\uline{This will result in a model that is simpler to apply, and
can also be more robust to variations in the parameters. }\textbf{\uline{(REF)}}


\subsection{Restricted Boltzmann Machine}

\cite{yu2011deep}:

A Restricted Boltzmann Machine (RBM) is a particular type of Markov
random field composed of one layer of binary stochastic hidden units
and another layer of stochastic visible (sometimes called \emph{observable})
units. Visible and hidden units aare densely connected by undirected
and weighted links. The joint probability distribution of the visible
units $v$ and hidden units $h$, given the model parameters $\theta$
of an RBM $p\left(v,h;\theta\right)$ is defined through an energy
function $E\left(v,h;\theta\right)$ such that

\[
p\left(v,h;\theta\right)=\frac{e^{-E\left(v,h;\theta\right)}}{Z}
\]


where $Z=\sum_{v}\sum_{h}e^{-E\left(v,h;\theta\right)}$ is a normalization
factor called the \emph{partition function} and the marginalized probability
that the model assigns to a visible vector $v$ is obtained by summing
over all possible hidden vectors 
\[
p\left(v;\theta\right)=\frac{\sum_{h}e^{-E\left(v,h;\theta\right)}}{Z}
\]


In the case where both the visible and hidden units are considered
binary (Bernoulli) variables, the energy function of the joint configuration
of $I$ visible and $J$ hidden units $(v,h)$ is defined as

\[
E\left(v,h;\theta\right)=\sum_{i=1}^{I}\sum_{j=1}^{J}w_{ij}v_{i}h_{j}-\sum_{i=1}^{I}b_{i}v_{i}-\sum_{j=1}^{J}a_{j}h_{j}
\]


where $w_{ij}$ is the symmetric weight (called \emph{interaction
term}) between visible unit $v_{i}$ and hidden unit $h_{j}$, and
$b_{i}$, and $a_{j}$ are the bias terms of these respective units.
The conditional probabilities that an hidden or visible unit is active
is given by 
\[
p\left(h_{j}=1\mid v;\theta\right)=\sigma\left(\sum_{i=1}^{I}w_{ij}v_{i}+a_{j}\right)
\]


\[
p\left(v_{i}=1\mid h;\theta\right)=\sigma\left(\sum_{j=1}^{J}w_{ij}h_{j}+b_{i}\right)
\]


where $\sigma\left(x\right)$ is the sigmo�d function. It is often
desirable (in typical real-life datasets) to consider the visible
units as real-valued instead of binary by relying on Gaussian units.
In this case, the RBM energy function is

\[
E\left(v,h;\theta\right)=\sum_{i=1}^{I}\sum_{j=1}^{J}w_{ij}v_{i}h_{j}+\frac{1}{2}\sum_{i=1}^{I}\left(v_{i}-b_{i}\right)^{2}-\sum_{j=1}^{J}a_{j}h_{j}
\]


The conditional probabilities of the visible units being active become

\[
p\left(v_{i}\mid h;\theta\right)=\mathcal{N}\left(\sum_{j=1}^{J}w_{ij}h_{j}+b_{i},1\right)
\]


Here the visible unit $v_{i}$ is considered real-valued by following
a Gaussian distribution with mean $\sum_{j=1}^{J}w_{ij}h_{j}+b_{i}$
and unit variance.

The probability that the network assigns to an input training vector
can be raised by lowering its energy, which amounts to adjusting the
weights and biases of various units. Hence, the goal of learning is
to lower the energy of an input while simultaneously raising the energy
of other data, especially those with low energies and which therefore
make a large contribution to the partition function. In other words,
the network tries to place high probabilities to the vectors that
are part of the input dataset and low probabilities to the vectors
that are not. The derivative of the log-likelihood $\mbox{log }p\left(v;\theta\right)$
of an input vector with respect to a weight is surprisingly simple.

\[
\frac{\partial\mbox{log }p\left(v;\theta\right)}{\partial w_{ij}}=\mathbb{E}_{data}\left[v_{i}h_{j}\right]-\mathbb{E}_{model}\left[v_{i}h_{j}\right]
\]


where $\mathbb{E}_{data}\left[v_{i}h_{j}\right]$ denotes the expectation
under the distribution of the training data and $\mathbb{E}_{model}\left[v_{i}h_{j}\right]$
is that same expectation under the distribution defined by the model.
This leads to a very simple learning rule for performing gradient
ascent over the log-probability of the training data. It is very easy
to get an unbiased sample of the expectation under the distribution
of the data $\mathbb{E}_{data}\left[v_{i}h_{j}\right]$ because there
are no direct connections between hidden units in an RBM. Unfortunately,
the expectation under the distribution of the model $\mathbb{E}_{model}\left[v_{i}h_{j}\right]$
is intractable to compute. However, Hinton et al. \cite{hinton2006fast}
proposed a Contrastive Divergence (CD) approximation to the gradient,
where $\mathbb{E}_{model}\left[v_{i}h_{j}\right]$ is replaced by
running the Gibbs sampler initialized at the data for a given number
of full steps. Interestingly, even though this Gibbs chain should
be runned to infinity, it appears that a single full step is sufficient
to obtain satisfactory results. A \textquotedblleft reconstruction\textquotedblright{}
is produced by setting each visible unit $v_{i}$ to 1 with the previously
defined probability given by $p\left(v_{i}=1\mid h;\theta\right)$.
The change in a weight is then given by 

\[
\Delta w_{ij}=\epsilon\left(\mathbb{E}_{data}\left[v_{i}h_{j}\right]-\mathbb{E}_{recon}\left[v_{i}h_{j}\right]\right)
\]


Regarding the update rule for biases, a simplified version of the
learning rule can be used where pairwise products are replaced by
the states of individual units instead. Even though these rules are
approximations, the training seems to work surprisingly well even
with a single step of CD. Interestingly, , have shown that these update
rules do not correspond to the gradient of any function. Nevertheless,
its success in many applications is also deeply tied to a careful
selection of its parameters and implementation, such practical rules
are provided in\uline{ }\cite{hinton2010practical}. 


\subsubsection{Different types of unit}

Even though the sigmo�d has been historically the most used type of
unit and that RBMs were disgned for logistic binary units, other types
of units can be set in the layers. The main goal of these other types
of unit is to handle data which might be ill-suited to binary logistic
visible units. Hence, the choice of unit should mainly depend on the
nature of input data and its underlying distribution.

\textbf{+ Relate to activation function (which is considered as a
probability in the case of RBMs)}

\textbf{+ Warning when talking about different types of unit to well
distinguish AE units (tanh / relu)}


\paragraph{Sigmo�d and softmax units}

The probability of a sigmo�d unit to be active is given by the logistic
sigmoid function of its input 
\[
p=\sigma(x)=\frac{1}{1+e^{-x}}
\]


As we can see, if we generalize this function to take into account
$N$ alternative states, we obtain the softmax unit 
\[
p_{j}=\frac{e^{x_{j}}}{\sum_{i=1}^{N}e^{x_{i}}}
\]



\paragraph{Gaussian units}

For real-valued data such as the commonly studied inputs of most machine
learning fields, binary units are a way too coarse approximation and
can lead to a poor reconstruction. A simple solution to account for
reals instead of binary numbers in the visible inputs is to introduce
linear units with independent Gaussian noise. In the case of RBMs,
the energy function becomes: 
\[
E(v,h)=\sum_{i\in vis}\frac{\left(v_{i}-a_{i}\right)^{2}}{2\sigma_{i}^{2}}-\sum_{j\in hid}b_{j}h_{j}-\sum_{i,j}\frac{v_{i}}{\sigma_{i}}h_{j}w_{ij}
\]
with $\sigma_{i}$ is the standard deviation of the visible Gaussian
unit $i$. It should be noted that the use of Gaussian units require
a smaller learning rate, as now there is no bound on the magnitude
of the output in the reconstruction (oppositely, binary units being
naturally bounded in value). Therefore, a component can become arbitrarily
large which will result in a very large learning signal emanating
from this single unit. Oppositely, the learning signal of binary units
lie in the $\left[-1,1\right]$ range, which makes them more stable.


\paragraph{Binomial units}

The simplest way to handle integer values between $0$ and $N$ is
to rely on $N$ separate binary units but tying them to share identical
weights and bias \cite{teh2001rate}. As all these copies share the
same parameters, the same input will result in the same activation
probability that can be summed to obtain an integer. Furthermore,
this probability can be computed only once for the whole set of copies.
The utmost advantage of relying on these weight-sharing constructs
to synthesize a new type of unit is that the underlying model and
mathematics of RBM remains unchanged.


\paragraph{Rectified linear units}

By extending the previous reasoning to a potentially infinite number
of copies, the sum of these shared probabilities tends to having a
closed form: 
\[
\sum_{i=1}^{\infty}\sigma\left(x-i+0.5\right)\approx\mbox{log}\left(1+e^{x}\right)
\]


where $x=vw^{T}+b$. It can be seen that this type of infinity of
binomial units behaves like a smoothed rectified linear unit.

\uline{Even though $log(1+e^{x})$ is not in the exponential family,
we can model it accurately using a set of binary units with shared
weights and fixed bias offsets. This set has no more parameters than
an ordinary binary unit, but it provides a much more expressive variable.}


\subsubsection{Conditional RBM}

The first extension of the RBM proposed to handle multivariate time-series
was the conditional RBM (cRBM) \cite{taylor2006modeling,taylor2009factored}.
The cRBM is constructed with the same architecture as an RBM, but
adding connections between a set of past visible vectors to the current
hidden units. These links can be seen as vectors of auto-regressive
weights that provide a form of short-term memory over the temporal
structures. This dependency over a time frame of $n$ past visible
units is modeled through the bias vectors of the cRBM defined as 
\begin{eqnarray*}
b_{i}^{*} & = & b_{i}+\sum_{k=1}^{n}B_{k}\mathbf{x}_{\left(t-k\right)}\\
c_{i}^{*} & = & c_{i}+\sum_{k=1}^{n}A_{k}\mathbf{x}_{\left(t-k\right)}
\end{eqnarray*}


where $A_{k}$ models the auto-regressive weights between a visible
units at $k$ previous time steps and the current visible units and
$B_{k}$ models the same relationship of the past visible but to the
current hidden units. The order of the model is defined by the length
of its memory, i.e. the number of previous time frames $n$ that are
taken into account. The activation probabilities of the hidden and
visible units are given respectively by

\begin{eqnarray*}
P\left(h_{j}=1\mid\mathbf{x}\right) & = & \sigma\left(b_{j}+\sum_{i}W_{ij}x_{i}+\sum_{k}\sum_{i}B_{ijk}x_{i}\left(t-k\right)\right)\\
P\left(x_{i}=1\mid\mathbf{h}\right) & = & \sigma\left(c_{i}+\sum_{j}W_{ij}h_{j}+\sum_{k}\sum_{i}A_{ijk}x_{i}\left(t-k\right)\right)
\end{eqnarray*}


The set of parameters $\theta=\left\{ W,b,c,A,B\right\} $ is trained
like a traditionnal RBM through contrastive divergenceand the cRBM
can also be used as a building block to create deeper networks. Conditional
DBN has been used for human motion analysis {[}\textbf{xREFx}{]}.
The conditional aspect associates the weight with a time window over
the data from previous time steps. This leads to a type of temporal
DBN improving the handling of temporal coherence aspects.


\subsubsection{Temporal RBM}

A similar model is the Temporal RBM \cite{sutskever2007learning}
which has been proposed as an extension to cRBM. The idea is to provide
context for the past visible units (same as the cRBM), but also to
embed context information for the hidden states as well. Hence, the
model is built of
\[
P\left(\mathbf{h}_{t},\mathbf{x}_{t}\mid\mathbf{h}_{t-1},\mathbf{x}_{t-1},\ldots,\mathbf{h}_{t-k},\mathbf{x}_{t-k}\right)
\]


where the context is defined for a window of $k$ time steps over
both the visible and hidden units. Even if sampling in the Temporal
RBM can be done in the same Markov chain approximation as in cRBMs,
the inference aspect becomes intractable. This problem can be resolved
by using a mean-field approximation instead of exact inference.


\subsubsection{Gated RBM}

The Gated RBM (GRBM) \cite{memisevic2007unsupervised} is another
extension of the RBM targeted at modeling temporal data by directly
incorporating the transitions between input vectors of consecutive
time frames. To do so, the GRBM introduces a weight tensor $W_{ijk}$
which represent the interaction between the input $x$, the output
$y$, and a set of latent variables $z$ (called \emph{transformation
variables}) which are considered the hidden units. The energy function
is defined as: 

\[
E\left(y,z;x\right)=-\sum_{ijk}W_{ijk}x_{i}y_{i}z_{j}-\sum_{k}^{I}b_{k}z_{k}-\sum_{j}c_{j}y_{j}
\]


Hence, the GRBM tries to model directly what are the statistical regularities
of going from one input to the next through the concept of transformation.
The conditional probability of this transformation is given by
\[
p\left(y,z\mid x\right)=\frac{e^{-E\left(y,z;x\right)}}{Z}
\]


with $Z$ the partition function and the probability that hidden unit
$z_{i}$ is active given an input $x$ and output $y$ is given by

\begin{eqnarray*}
P\left(z_{k}=1\mid x,y\right) & = & \sigma\left(\sum_{ij}W_{ij}x_{i}y_{j}+b_{k}\right)
\end{eqnarray*}


Each hidden variable $z_{k}$ learns an aspect of the transformation
between the input $x$ and the output $y$. Hence, for a fixed input,
the consecutive time frame creates a RBM learning the transformation
that could produce this next output. This type of learning could not
be achieved by simply concatenating two time frames and feeding it
to a regular RBM since the latent variables would only thrive on the
statistical regularities between that particular pair and not learning
the general transformation.


\subsubsection{Factored RBM}

The Factored RBM \cite{mnih2007three} is a parametrization of RBM
proposed to learn a distributed representation of words.


\section{Deep architectures}


\subsection{Stacked auto-encoders}

The DAE and RBM models are very shallow network in which a single
layer of computation (given by the hidden layer activations) provides
the set of learned features. However, the aim of deep neural networks
is to obtain multiple hidden layers with each providing increasingly
higher-level and complex features over the input by uncovering correlations
through non-linear transformation of the previous layer. One of the
key aspect in the expressive power of deep networks lie in the use
of non-linear activation functions in various hidden layers. Indeed,
as a combination of multiple applications of linear functions is itself
only a linear function of the input, a network relying only on linear
computations units could simply be reduced to a single layer of linear
hidden units.

The representational power of deep networks is based on its ability
to learn a structured hierarchy. This approach is based on the underlying
hypothesis that higher-level concepts can be formed by grouping lower-level
ones. Hence, each layer is expected to form increasingly complex features
by learning how to optimially group concepts formed at the previous
layer, by exploiting the statistical regularities of lower-level concepts.

Based on these observations, we see that we could form a deep network
by simply stacking autoencoders on top of each other and feeding the
latent representation of the auto-encoder at a specific layer as input
to its above layer. In that case, the stacked autoencoder can either
be interpreted as a list of separate autoencoders, or more globally
as a traditional MLP. By relying on this type of dichotomy, we see
that unsupervised pre-training can be done one layer at a time, by
training each layer as an auto-encoder minimizing the reconstruction
of the hidden representation output by the previous layer. After training
the first $k$ layers, we can train the layer $k+1$ by computing
the hidden representation obtained by iteratively passing the input
data to all the $k$ layers below. After all the layers have been
trained iteratively, the global view of the network allows to consider
a second stage of training called \emph{fine-tuning}, usually by minimizing
the error rate of a supervised task. In that case, a logistic regression
(classification) layer can be added on top of the network in order
to rely on the highest-level representation provided by the network
(hidden activations of the last layer). The training procedure of
the entire network is then exactly similar to that of a traditional
MLP.

Following the idea of self-taught learning, features can be learned
using only unlabeled data. However, it is also possible to combine,
fine-tune and further improve these unsupervised features using labeled
data. The overall classifying architecture can simply be considered
as a deeper neural network.


\subsubsection{Pre-training stacked autoencoders}

By relying on a greedy layerwise pretraining, a deep stacked autoencoder
can be formed from a set of $k$ autoencoders, each parametrized by
its own weight matrix $\mathbf{W}_{e}^{(l)}$ and biases $b_{e}^{(l)}$.
Therefore, the output (activation) of the stacked autoencoder at each
layer is produced by iteratively performing a forward pass of the
encoding step from each layer to the next

\begin{eqnarray*}
a^{(l)} & = & \phi\left(\mathbf{W}_{e}^{(l-1)}\mathbf{a}^{(l-1)}+b_{e}^{(l-1)}\right)
\end{eqnarray*}
by considering that the first layer receives the output $\mathbf{a}^{(0)}=\mathbf{x}$.
Conversely, the decoding step can be obtained by performing the decoding
of each autoencoder in reverse order. The stacked auto-encoder ultimately
produces a representation $a^{(n)}$, which is the activations of
the deepest hidden units. This representation vector can be seen as
a collection of highest-order features extracted from the input, which
can in turn be used for classification problems with a softmax classifier.


\subsubsection{Fine-tuning stacked auto-encoders}

As each layer of autoencoders is built on the output of the previous,
the stacked autoencoder can be considered as a single model formed
by a complete network. This global model could, therefore, be trained
altogether by improving upon the weights of all layers at each iteration.
This\emph{ finetuning} operation can be performed by discarding the
decoding layers and replacing them by linking the last hidden layer
to a softmax classifier. The error gradients from the supervised classification
mistakes can then be backpropagated into all the encoding layers together.

As the backpropagation algorithm can be applied to a network of arbitrary
depth, we can actually rely on the gradients from the final classification
layer and backpropagate them across all the encoding layers. The only
change to apply is to adapt the back-propagation step of the last
layer to fit a softmax evaluation

\[
\delta^{(n_{l})}=-(\nabla_{a^{n_{l}}}J)\bullet f'(z^{(n_{l})})
\]


\uline{(When using softmax regression, the softmax layer has} $\nabla J=\theta^{T}(I-P)$
\uline{where $I$ is the input labels and $P$ is the vector of
conditional probabilities.)}


\subsection{Deep Belief Networks}

Deep Belief Networks (DBNs) \cite{hinton2006fast} are a class generative
models formed by multiple layers of stochastic computation units.
The lowest layer (usually called the \emph{visible layer}) is composed
of units, whose state (their latent variable) represent an input vector.
The layers are connected with directed top-down connections from above
layer and are called the \emph{hidden units }(also called \emph{feature
detectors}). The two upper layers are created with a dense array of
undirected symmetric connections, which form an associative memory.
Hence a $l$-layers DBN models the joint distribution between all
hidden layers $\mathbf{h}^{k}$ and the observed data $\mathbf{x}$
\[
P\left(\mathbf{x},\mathbf{h}^{1},\ldots,\mathbf{h}^{k}\right)=\left(\prod_{k=0}^{(l-2)}P\left(\mathbf{h}^{k}\mid\mathbf{h}^{k+1}\right)\right)P\left(\mathbf{h}^{l-1},\mathbf{h}^{l}\right)
\]


with $\mathbf{h}^{0}=\mathbf{x}$. We can see the construct of DBN
through these two terms, where the lower layers are directed and the
last one is formed as an RBM with undirected connections. We can see
that stacking layers of RBMs (learned iteratively in a greedy layer-wise
training) from the lowest (\emph{visible }data) to the upper (\emph{associative
memory}) layer can provide a DBN architecture. In that case, we consider
that the activation probabilities of the hidden layer of one RBM becomes
the visible data for the next RBM layer. It has been shown that relying
on this stacking procedure improves the variational lower bound of
the log-likelihood of the data \cite{hinton2006fast}. This means
that learning a DBN with this procedure provides a close approximation
of the true maximum likelihood (ML) learning. 

This pre-training and learning phase of the DBN is usually followed
by the addition of a discriminative layer in order to perform discriminative
task. This allows to \emph{fine-tune }the whole networks by adjusting
the complete sets of weights jointly in order to further improve the
accuracy of the model. The advantages of the DBN is that it can be
interpreted in the Bayesian framework as a probabilistic generative
model, which allows both to efficiently compute the hidden variables,
but also to sample from the network to synthesize new data \cite{yu2011deep}.

Tang and Eliasmith showed that sparse connection patterns in the first
layer of the DBN and a probabilistic denoising algorithm could improve
the robustness of the DBN \cite{tang2010deep}.

\textbf{Extend this + FIGURE}


\subsection{Deep Boltzmann Machine}

The joint training of layers in a DBN is problematic as the implied
inference problem is often intractable. The Deep Boltzmann Machine
(DBM) has been proposed to allow a joint training of all layers of
deep architectures in a purely unsupervised manner \cite{salakhutdinov2009deep}.
Similarly to the RBM, a DBM is a specific type of Boltzmann machine
wirh layered hidden units. However, the DBM is composed of multiple
layers, in which conditionnal independence is imposed between odd-numbered
layers and even-numbered layers (still without intra-layer connections).
This allow DBM to directly learn increasingly complex representations
but also resorting directly to massive amounts of unlabeled data.
However, unlike DBN, their inference procedure can incorporate top-down
feedback (because of undirected connections between layers), which
allow to perform a better propagation of uncertainty.

A 2-layer DBM (with one visible layer and two hidden layers $\left\{ \mathbf{v},\mathbf{h}^{1},\mathbf{h}^{2}\right\} $
and parameters $\theta$) can be defined by specifying its energy
function
\[
E\left(\mathbf{v},\mathbf{h}^{1},\mathbf{h}^{2};\theta\right)=-\mathbf{v}^{T}\mathbf{W}^{1}\mathbf{h}^{1}-\mathbf{h}^{1T}\mathbf{W}^{2}\mathbf{h}^{2}
\]


Hence, the probability that the model assigns to a visible vector
is given by
\[
p\left(\mathbf{v},\theta\right)=\frac{1}{Z}\sum_{\mathbf{h}^{1},\mathbf{h}^{2}}\mbox{exp}\left(-E\left(\mathbf{v},\mathbf{h}^{1},\mathbf{h}^{2};\theta\right)\right)
\]


Even though the the conditional distributions over the visible units
and the last hidden units are defined in a similar way than in RBM,
the distribution over the ``middle'' layer of hidden units is defined
as
\[
p\left(h_{j}^{1}\mid\mathbf{v},\mathbf{h}^{2}\right)=\sigma\left(\sum_{i}W_{ij}^{1}v_{i}+\sum_{m}W_{jm}^{2}h_{j}^{2}\right)
\]


Hence, the major problem with DBM (as compared to RBM) is that these
interactions between hidden units make the posterior of hidden units
untractable. To solve this problem, \cite{salakhutdinov2009deep}
propose to rely on a mean-field approximation (similar to the ideas
of \emph{variational learning}), where the posterior distribution
$P\left(\mathbf{h}^{1},\mathbf{h}^{2}\mid\mathbf{v}\right)$ is approximated
with a factored distribution $Q\left(\mathbf{h}^{1},\mathbf{h}^{2}\right)=\prod_{j}Q\left(h_{j}^{1}\right)\prod_{i}Q\left(h_{i}^{2}\right)$
such that the KL divergence between the real and approximated distributions
$KL\left(P\left(\mathbf{h}^{1},\mathbf{h}^{2}\mid\mathbf{v}\right)\parallel Q\left(\mathbf{h}^{1},\mathbf{h}^{2}\right)\right)$
is minimized. This is equivalent to maximizing a lower on the log-likelihood
\[
\mathcal{L}\left(Q\right)=\sum_{h^{1}}\sum_{h^{2}}Q\left(\mathbf{h}^{1},\mathbf{h}^{2}\right)\mbox{log}\left(\frac{P\left(\mathbf{h}^{1},\mathbf{h}^{2}\mid\mathbf{v}\right)}{Q\left(\mathbf{h}^{1},\mathbf{h}^{2}\right)}\right)
\]


In order to maximize this lower-bound, developping the zero derivatives
with respect to the approximated distribution $Q\left(\mathbf{h}^{1},\mathbf{h}^{2}\right)$,
we obtain the update equations
\[
h_{j}^{1}\leftarrow\sigma\left(\sum_{i}W_{ij}^{1}v_{i}+\sum_{m}W_{jm}^{2}h_{m}^{2}\right)
\]


The training phase is proposed through a variational procedure where
the positive phase is modified but the negative phase can be estimated
through contrastive divergence, similarly to RBM.


\subsection{Temporal models}

Unsupervised feature learning and deep learning algorithms were initially
developed for static data and features. Hence, in order to apply these
methods to datasets with a clear temporal nature, these approaches
should be adapted to fit the subtelty and challenges raised by time
series datasets \cite{langkvist2014review}. 

Time-series data are a peculiar object of study with several characteristics
that distinguish them from other types of data. Firstly, the most
prominent problems arise from the high dimensionality of time series
data. This lead to the problem known as the \emph{curse of dimensionality
}which prevents traditional learning algorithms to decipher the inner
correlations of temporal data. Second, time-series data often stem
from a sampling process which induces noise in the observation. To
alleviate this problem, signal processing techniques such as low-pass
filtering or spectral analysis can be applied to remove some of this
high-frequency noise. Third, the explicit dependency on the temporal
dimension entails a notion of temporal granularity in which the time
dependencies can coexist at various scales. Finally, \uline{there
is a difference between time-series data and other types of data when
it comes to invariance. Most features used for time-series need to
be invariant to translations in time}. \textbf{+ REPLACE THIS WITH
MY AXIOMS OF ROBUSTNESS}

Different time-series datasets will usually exhibit different degrees
of these peculiarities, which warrants the need of incorporating prior
knowledge on these characteristics in the chosen learning framework.
Hence, the feature extraction mechanisms have to be modified in order
to capture temporal dependencies by adjusting to the properties of
temporal data.

Real-world time-series data are often high-dimensional, noisy and
pertain to generating mechanisms that cannot be modeled through analytical
equations since their dynamics are either unknown or too complex to
be formalized. \uline{it is not certain that there are enough information
available to understand the process. }\textbf{\uline{NB: What was
the rule in the line that ``we need at least as many examples as
there are dimensions of variation''.}}\uline{ Many time-series
are also non-stationary, meaning that the characteristics of the data,
such as mean, variance, and frequency} vary across different windows
of the series\uline{. }\textbf{\uline{PARAPHRASE + USE OPPOSITION
WITH THE STATIONARITY HYPOTHESIS}} Hence, shallow methods typically
applying a single layer of non-linear operations seem ill-suited to
accurately model data of such complex nature.


\subsubsection{Recurrent Neural Network}

Recurrent Neural Network (RNN) \cite{pineda1987generalization} have
been used for modeling temporal data long before the advent of deep
learning. An RNN can simply be obtained by modifying a feedforward
network in order to allow ``loops'' where the output of neurons
are connected to their own inputs. These looped connections allow
to model the short-term time-dependencies without using any time delay-taps.
In order to solve the propagation indeterminacy when trying to update
the weights in training phase, an iterative algorithm such as the
backpropagation-through-time (BPTT) \cite{werbos1990backpropagation}
can be used. Alternative strategies for training RNNs can also be
employed \cite{sutskever2007learning}. The input is transformed through
the hidden units that have connections between input of the current
time frame and output from the previous time frame (modeled by the
loops in the connections). \uline{A popular extension is the use
of the purpose-built Long-short term memory cell \mbox{\cite{hochreiter1997long}}
that better finds long-term dependencies.}


\subsubsection{Convolution and pooling}

Convolution offers a particularly interesting framework for high-dimensional
data, such as time-series data. The fact that in convolutional networks,
units in the hidden are \emph{locally }connected to contiguous visible
segments (instead of being fully connected) both provides a natural
way to handle temporal continuity and an increased computational efficiency.
Both the AE and RBM models have been extended to incorporate convolutional
operators in order to produce convolutional AEs (convAE) \cite{masci2011stacked}
and convolutional RBMs (convRBM) \cite{lee2009convolutional}. Classical
neural networks have been specialized to exploit the input time structure
by performing convolutions on overlapping windows in an approach called
Time-Delay Neural Network (TDNN) \cite{waibel1989modular}.

Along with convolution, which creates high-dimensional replicates
of the inputs through different feature extractors, the pooling operation
can combine locally contiguous input values or features through the
application of an average, max or histogram operator. Pooling not
only provides invariance to small distortions in the local neigborhood
but also drastically reduces the dimensionality of the feature space.
However, the major drawback of pooling is that it is non-differentiable.
This can be alleviated by relying on the \emph{probabilistic max-pooling
}operator introduced by Lee et al. \cite{lee2009convolutional}. Another
proposed approach is the Space\textendash Time DBN (ST-DBN) \cite{chen2010deep}based
on convolutional RBMs in which a separate pooling layer is applied
to the spatial and temporal dimensions separately to build spatio-temporally
invariant features.


\subsubsection{Temporal coherence}

Temporal coherence is one of the most important property of time series
data and can be related to invariant feature representations. The
goal in this case would be that small changes in the input data (contiguous
time frames) only incur small changes in the feature representation,
which can also be achieved using a structured form of sparsity penalty
targeted at temporal relationships \cite{kavukcuoglu2009learning}.

Hence, asides from the previously introduced various architectural
constructs that can be used to capture temporal dependencies, various
types of smoothness penalties on the hidden variables can be introduced
to enforce temporal regularization. This is usually done by penalizing
the squared differences in the hidden unit activations between contiguous
time frames. The main idea behind this type of penalty is that sequential
data is supposed to be smooth, so the hidden activations should not
vary much if this temporal data is fed to the model in a chronological
order.


\subsubsection{Comparison}

The major difference between models targeted at temporal data analysis
lies in their implementation of the notion of temporal memory. In
a cRBM, the short-term dependencies across visible units are modeled
through delay taps and the longer-term dependencies are a by-product
of subsequent layers modeling. Hence, the length of a cRBM memory
can be increased by increasing the number of layers accounted in the
model. In an RNN, the loop connections causes hidden units to be influenced
by their state in the previous time frame. These connections can can
create a \emph{ripple} effect, which can last over a potentially infinite
number of time frames. This ripple effect can be prevented through
the addition of a forget gate \cite{gers2000learning} which periodically
removes the effect of the feedback connection. By relying on Hessian-free
optimizer \cite{martens2012training} or the Long-short term memory
\cite{hochreiter1997long} models, recurrent networks can provide
a longer-term memory spanning across more than a hundred time steps.
\uline{The Gated RBM and the convolutional GRBM models transitions
between pairs of input vectors so the memory for these models is 2}.


\subsubsection{Summary}

Based on the presented models, a number of problem-specific recommandations
should be raised to investigate temporal data analysis. Hence, selecting
the best-suited model for a specific problem first requires to take
several questions into considerations
\begin{enumerate}
\item \emph{The characteristics and underlying structure of the data}. The
type of pre-processing, choice of the models and even their parametrization
should heavily depend on the inner characteristics of the studied
data. If the data has an inherent dimension, the typical feature vector
approach would discard any temporal relationship. Instead, a model
targeted at exploiting temporal regularities through either notions
of memory (by architectural structure) or temporal coherence (by regularization
penalties) has a higher chance of providing satisfactory results.
\item \emph{Relying on generative or discriminative models}. A generative
model provides a straightforward way of synthesizing new data but
also predicting partial input data that would need to be reconstructed.
Generative models are, therefore, usually more robust to noisy inputs
and provide a better detection of outliers. On the other hand, discriminative
models are more efficient for classification problems and also easier
to implement.
\item \emph{The dimensionality of the input data}. For large-scale problems,
stochastic gradient descent can provide a faster optimization method
\cite{bottou2010large}. Massive datasets can also precludes the use
of longer-term memory models and appropriate pre-processing methods
should be applied.
\end{enumerate}
The premise of deep networks is that a structured hierarchy of abstractions
can be learned from the underlying distribution of data. This hypothesis
fits the construct of musical data in which notes group into chords,
temporally arranged to create melodies and rhythms which in turn makes
motifs and phrases emerging to construct entire musical pieces \cite{humphrey2013feature}.
Hence, deep learning algorithms could target these elementary building
blocks (musical motifs) and complex musical pieces could be constructed
from a hierarchical structure of these previously learned motifs.

\uline{Even though convolutional networks have given good results
on time-frequency representations of audio, there is room for discovering
new and better models.}

Deep networks can, in an unsupervised manner, learn these motion templates
from raw data and use them to form complex human motions.

Constructing features learning from raw data has been extensively
studied in vision tasks but is yet to be attempted in music recognition.
\uline{Models such as TDNN, cRBM and convolutional RBMs are well
suited for being applied to raw data, but most works that construct
useful features from the input data actually still use input data
from pre-processed features}


\section{Other models}


\subsection{Convolutional Neural Networks}

Convolutional Neural Networks (CNN) are variants of Multi-Layer Perceptrons
(MLPs) inspired by the works on the architecture of the visual cortex
\cite{hubel1968receptive}, where the cells seem to be sensitive only
to small sub-regions of the visual field. These regions, called \emph{receptive
fields} are targeted to exploit the local spatial regularities by
acting as local filters over the input. Hence, each simple cell will
respond maximally to a local region of space within this receptive
field. Oppositely, more complex cells have a response to larger spatial
regions and appear to provide an amount of spatial invariance to position
and transformation of the patterns. These ideas were developed in
convolutional form of neural networks such as LeNet \cite{lecun1998gradient}.


\subsubsection{Sparse connectivity}

Traditionnal neural networks are built on fully connected layers,
where all hidden units are linked to all input units, which can imply
a very high computational cost when working on large datasets with
high dimensionalities. However, as proposed by the work on the visual
cortex, some neurons seem to target only local regions, to exploit
the strong spatially-local correlation present in an image. These
ideas could be easily reflected by restricting the connections of
hidden units and enforcing local connectivity patterns between the
input and hidden units. Hence, each hidden unit in one layer would
only be connected to a small contiguous subset of units from the layer
below. This mechanism mimics the idea that these neurons only focus
on a local region of spatially contiguous information.

This type of sparse connectivity ensure that the units only respond
to variations inside their own receptive field. Therefore, the ``filters''
learned are only sensitive to spatial locality. Furthermore, this
strongly reduces the number of weights and parameters to be learned,
which enhance the learning efficiency of the network.

\textbf{+ FIGURE}


\subsubsection{Convolutions}

CNNs exploit the property of \emph{stationnarity}, in which the underlying
statistics of one part of the input are equivalent to the statistics
of any other part. This property imply that a feature learned at a
specific location will be useful for any other location. Therefore,
the same local features learned from random subsets of the complete
input can be applied at all the locations in the input. This operation
is similar to the \emph{convolution }operator, by applying this local
feature (seen as a \emph{kernel}) over each location of the input.

More formally, if we learn a set of $k$ features from local connectivity
sub-parts of size $s_{x}\times s_{y}$, we can convolve this feature
over the complete input of size $l_{x}\times l_{y}$ to obtain an
array of convolved features of size $k\times(l_{x}-s_{x}+1)\times(l_{y}-s_{y}+1)$.


\subsubsection{Shared weights}

Based on this construct, we can see that each local filter will be
applied across the entire input. This lead to a form of replication
where a set of units perform the same operation at different locations.
Therefore, these units have the same set of parameters (weights and
bias) which lead to a form of \emph{weight sharing}. In that case,
the gradient can be computed efficiently only once for a unit and
then summed across the number of replicated units.

This replication leads to increasing the robustness of the network
to variance in position of the input, but also greatly improves the
learning efficiency of the network by reducing its number of parameters.


\subsubsection{Feature maps}

The replication of local filters applied across all regions of the
input leads to a convolved array of features called a \emph{feature
map}. More formally, given one local filtering unit $k$ parametrized
by its weights $W_{k}$ and bias $b_{k}$, the feature map $h^{k}$
is obtained by applying the non-linear activation function to the
local application of the feature to a specific location $\left(i,j\right)$
in the input 

\[
h_{ij}^{k}=\phi\left(\left(W_{k}*x\right)_{ij}+b_{k}\right)
\]


In order to obtain multiple features from the image (and, therefore
a more detailed representation of the data), the hidden layers are
composed of multiple feature maps $\left\{ h^{k},k=0\ldots N\right\} $.
Each of these feature maps provide the value of the learned features
at all points in the input.

\textbf{+ FIGURE + Pooling figure}


\subsubsection{Max pooling}

These features maps extracted using convolution could theoretically
be used altogether directly as input to a classifying layer such as
a softmax classifier. However, given the obtained replication and
largely increased dimensionality this can turn out to be computationally
challenging. However, the original hypothesis of \emph{stationarity}
still holds and implies that features are likely to be similarly useful
in any region of the input.

Based on these properties, it would seem logical to aggregate the
values of the features across different positions in the global input.
This would summarize the presence of the different features in increasingly
large regions of the input. This form of down-sampling operation is
called \emph{pooling}, which can be performed through the average
(mean-pooling) or maximum (max-pooling) of the values inside a region.
This operation partitions the input into a set of (non-overlapping)
regions and outputs the corresponding value at each corresponding
sub-region.

The pooling operation provides many attractive properties. First,
as it eliminates the non-maximal values, it reduces the dimensionality
and, therefore, also reduces the required computations in upper layers.
Second, as the activation of a maximal value will remain identical
with slight shifts (in the dimension of the contiguous region), it
provides a degree of translation invariance. Hence pooling can be
seen as an efficient non-linear dimensionality reduction and invariance
technique.


\subsubsection{Tying the full model together}

By stacking several layers of these local non-linear filters we can
thus construct a deep network. As each layer is locally connected
to the layer below, it can be seen as a form of local grouping. Therefore,
the complete network provides increasingly global processing.

In order to tie the model together, lower-layers will be composed
by alterning sparsely connected layers (performing convolution and
producing the features map) and pooling layers (that regroup these
features into regions). Based on this increasingly global analysis
of various features, the final layers can be devised as a traditionnal
MLP, with densely connected units aimed at performing discrimination.

\textbf{+ FIGURE}


\subsubsection{Choosing hyperparameters}

Even though CNNs provide a more expressive and versatile learning
framework, they also imply a far larger number of hyper-parameters
than a standard MLP. Hence, they require an amount of caution and
experience when selecting their parameters.


\paragraph{Filters}

The CNNs are heavily conditionned on the number of filters used at
each layer. A quite logical thinking would be to assume that the more
features are computed, the more equipped the network will be to deal
with any type of situation. However, a single convolutional layer
is not only strongly more expensive to compute (in terms of forward
activations) than usual units, but also demultiplies the dimensionality
of the input by the number of filters used. Hence, for a layer of
$K$ features with an input of size $l_{x}\times l_{y}$ and filters
of size $s_{x}\times s_{y}$, this layer will require $k\times s_{x}\times s_{y}\times(l_{x}-s_{x})\times(l_{y}-s_{y})$
operations. Therefore, the number of filters is usually picked so
that this number of operations is almost equivalent to that of a traditionnal
sigmoid layer. It should be noted however that this implies a trade-off
between the capacity of the network and its computational complexity.

The shape of filters (in terms of dimensions) can also widely impact
performances. This usually depends on the nature of the dataset at
hand. The ideal filter shape is conditionned by a form of \emph{granularity
}depending on the amount and extent of spatially local correlations.
Hence, the type of abstractions that are sought by the network should
directly influence the scale of the filters.


\paragraph{Pooling}

The pooling operation defines the amount of invariance provided by
the network. However, it should be reminded that this is also a form
of lossy compression that will strongly reduce the dimensionality
of the input. This will provide computational efficiency but at the
cost of loosing part of the exact locational information. Regarding
the pooling function, it is widely accepted that the \emph{max }operation
appears to be the most robust as compared to the average.


\subsection{Sparse coding}

The aim of sparse coding \cite{olshausen1997sparse} is to find a
decomposition of an input vector $\mathbf{x}$ as being a linear combination
of a set of basis vectors $\mathbf{\phi}_{i}$, weighted by a set
of activation weights $a_{i}$ 
\[
\mathbf{x}=\sum_{i=1}^{k}a_{i}\mathbf{\phi}_{i}
\]


Simlarly to autoencoders, we wish to learn a set of basis vectors
which is over-complete. This can be understood as the goal to better
capture a wider set of higher-level structures and patterns underlying
to the distribution of input data. However, this over-completeness
also introduces the risk of a degeneracy in which nothing prevents
the network from learning the identity function. Therefore, the additional
constraint of sparsity is introduced to alleviate this risk and the
sparse coding cost function on a set of m input vectors becomes 
\[
\text{minimize}_{a_{i}^{(j)},\mathbf{\phi}_{i}}\sum_{j=1}^{m}\left|\left|\mathbf{x}^{(j)}-\sum_{i=1}^{k}a_{i}^{(j)}\mathbf{\phi}_{i}\right|\right|^{2}+\lambda\sum_{i=1}^{k}S(a_{i}^{(j)})
\]


where $S:\mathbb{R}\rightarrow\mathbb{R}$ is a sparsity penalty which
prevents the $a_{i}$ to be far from zero. Even though the most natural
sparsity measure would be to use the $L^{0}$ norm ($S(a_{i})=\mathbf{1}(|a_{i}|>0)$),
it is non-differentiable and the $L^{1}$ ($S(a_{i})=\left|a_{i}\right|_{1}$)
or log ($S(a_{i})=\log(1+a_{i}^{2})$) penalties are preferred. Contrarily
to the AE or RBM cost where the reconstruction is bounded by the direct
use of the input, sparse coding allow to optimize simultaneously the
basis vectors $\phi_{i}$ and the activation $a_{i}$. This also imply
that the sparsity penalty could be artificially small by scaling the
$\phi_{i}$ by a large constant, which would arbitrarily scale down
the $a_{i}$ (and therefore the sparsity). To prevent this behavior,
we need to constraint the magnitude of the basis vectors $\left|\left|\mathbf{\phi}\right|\right|^{2}$
to remain under a constant $C$, which leads to the complete cost
function\uline{ }
\[
\begin{array}{rc}
\text{minimize}_{a_{i}^{(j)},\mathbf{\phi}_{i}} & \sum_{j=1}^{m}\left|\left|\mathbf{x}^{(j)}-\sum_{i=1}^{k}a_{i}^{(j)}\mathbf{\phi}_{i}\right|\right|^{2}+\lambda\sum_{i=1}^{k}S(a_{i}^{(j)})\\
\text{subject to} & \left|\left|\mathbf{\phi}_{i}\right|\right|^{2}\leq C,\forall i=1,...,k
\end{array}
\]



\subsubsection{Probabilistic interpretation}

By interpreting the sparse coding in a probabilistic framework, we
see that its goal is that the distribution $P\left(\mathbf{x}\mid\mathbf{\phi}\right)$
of the reconstruction of the input $\mathbf{x}$ given a set of basis
vectors $\phi$ is closest to the distribution of the data $P\left(\mathbf{x}\right)$.
This goal can be achieved by minimizing the KL divergence between
the two distributions $KL\left(P\left(\mathbf{x}\mid\mathbf{\phi}\right)\parallel P\left(\mathbf{x}\right)\right)$.
To obtain the distribution of the reconstruction $P\left(\mathbf{x}\mid\mathbf{\phi}\right)$
we first need to define the prior distribution $P\left(\mathbf{a}\right)$
that can be factorized by assuming independence of source features
\[
P(\mathbf{a})=\prod_{i=1}^{k}P(a_{i})
\]


Then the probability of the data can be written under the reconstruction
performed by $\phi$ and $\mathbf{a}$ such that\uline{ }
\[
P(\mathbf{x}\mid\mathbf{\phi})=\int P(\mathbf{x}\mid\mathbf{a},\mathbf{\phi})P(\mathbf{a})d\mathbf{a}
\]


Finally, the sparse coding problem is to find the set of basis vectors
that maximizes this probability, which is usually performed using
the log-likelihood

\[
\mathbf{\phi}^{*}=\text{argmax}_{\mathbf{\phi}}\left(\mathbb{E}_{\mathbf{x}}\left[\log(P(\mathbf{x}\mid\mathbf{\phi}))\right]\right)
\]


with $\mathbb{E}_{\mathbf{x}}$ defining the expectation over the
data. Given this log-likelihood definition (which is equivalent to
minimizing the energy), we can obtain the original formulation of
the problem\uline{ }
\[
\mathbf{\phi}^{*},\mathbf{a}^{*}=\text{argmin}_{\mathbf{\phi},\mathbf{a}}\sum_{j=1}^{m}\left|\left|\mathbf{x}^{(j)}-\sum_{i=1}^{k}a_{i}^{(j)}\mathbf{\phi}_{i}\right|\right|^{2}+\lambda\sum_{i=1}^{k}S(a_{i}^{(j)})
\]


An interesting variation of sparse coding is the \emph{Spike-and-Slab
Sparse Coding }(S3C) \cite{goodfellow2012large} proposed for feature
learning. This model adds a set of latent \emph{spike }binary variables
and a set of \emph{slab }real-valued variables, where the activations
of \emph{spikes }directly controls the sparsity of the model.


\subsubsection{Autoencoder interpretation}

Sparse coding can be interpreted as a sparse autoencoder in which
both the set of sparse features useful for representing the data,
and the basis for projecting from the feature space to the data space
are learned simultaneously. Within this framework, the objective function
can be written as
\[
\mathcal{J}(A,s)=\left\Vert As-x\right\Vert {}_{2}^{2}+\lambda\left\Vert s\right\Vert {}_{1}
\]


where $\left\Vert \mathbf{x}\right\Vert {}_{k}=\left(\sum\left|x_{i}^{k}\right|\right)^{\frac{1}{k}}$
refers to the $L^{k}$ norm of the vector $\mathbf{x}$. Because of
the previously mentionned problem of potential scaling on the sparsity
constraint, the additional constraint to ensure the scale of $A$
expressed as $A_{j}^{T}A_{j}\le1$ is required. The complete sparse
coding problem is expressed ass

\[
\begin{array}{rcl}
{\rm minimize} & \left\Vert As-x\right\Vert {}_{2}^{2}+\lambda\left\Vert s\right\Vert {}_{1}\\
{\rm s.t.} & A_{j}^{T}A_{j}\le1\;\forall j
\end{array}
\]


As we can see, if we consider that $A$ is fixed, the problem of finding
$s$ that minimizes the cost $\mathcal{J}(s)$ is convex. Reciprocally,
if $s$ is fixed, the problem of minimizing $\mathcal{J}(A)$ is also
convex. This points to a potential resolution algorithm that could
alternately optimize one of these two parameters while considering
the other fixed (optimize $A$ for a fixed $s$, then optimize $s$
for a fixed $A$). To obtain this simple optimization method, we must
however handle the additionnal scaling constraint on $A$. Hence,
this constraint is often weakened to a term similar to the idea of
\emph{weight decay }penalty that constrains the values of A to remain
small. This leads to the objective function
\[
\mathcal{J}(A,s)=\left\Vert As-x\right\Vert {}_{2}^{2}+\lambda\left\Vert s\right\Vert {}_{1}+\gamma\left\Vert A\right\Vert {}_{2}^{2}
\]


As the $L^{1}$ norm used in the sparsity constraint contains a discontinuity
at 0, it is not differentiable at 0 which is problematic for gradient-based
methods. Hence, it is usually smoothed out by using the approximation
$\left|x\right|\approx\sqrt{x^{2}+\epsilon}$ , where $\epsilon$
is considered as a very small value called \emph{smoothing parameter}.
The final objective is then 
\[
\mathcal{J}(A,s)=\left\Vert As-x\right\Vert {}_{2}^{2}+\lambda\left(\sum_{k}\sqrt{s_{k}^{2}+\epsilon}\right)+\gamma\left\Vert A\right\Vert {}_{2}^{2}
\]


This final formulation of the objective function can be optimized
iteratively, by alternating between finding $s$ that minimizes $\mathcal{J}\left(A,s\right)$
for a fixed $A$ and then finding $A$ that minimizes $\mathcal{J}\left(A,s\right)$
for a fixed $s$. The procedure is
\begin{enumerate}
\item Perform a random initialization of $A$
\item Repeat until convergence 

\begin{enumerate}
\item Find through gradient descent the $s$ that minimizes $J(A,s)$ given
the current $A$
\item Analytically solve the $A$ that minimizes $J(A,s)$ for the current
$s$
\end{enumerate}
\end{enumerate}
However, simply applying this algorithm as it is will usually not
produce satisfactory results, as two main tricks are required to achieve
better convergence. First, as each iteration takes a long time before
convergence, the algorithm is instead run at each iteration solely
on a \emph{mini-batch} (a different randomly chosen subset of the
dataset).

Second, the random initialization of the basis vectors $s$ at each
iteration might lead to slow convergence (as it might be initialized
in a sub-optimal region of the space). This problem can be alleviated
by performing a data-sensitive initialization of s
\begin{enumerate}
\item First, setting $s=W^{T}\mathbf{x}$ with $\mathbf{x}$ being the set
of input vectors.
\item Then, for each input dimension, we divide it by the norm of the corresponding
(i.e. for the same dimension) basis vector.
\end{enumerate}
This type of initialization attempts to start by finding a better
approximation of $s$ (by applying the reverse transform from $Ws\approx x$).
Then, the second step normalize the initialization to ensure that
the sparsity penalty is kept small.


\subsubsection{Topographic sparse coding}

It might be desirable to learn an \emph{ordered} set of features meaning
that adjacent neurons detect similar features (which mimics the actual
layout of biological neurons in the brain), leading to a notion of
\emph{topographic ordering}. Intuitively, if feature detectors are
laid out in neighborhood depending on their similarities, then activation
of a feature will imply an activation of its neighbor (to a lesser
extent). Hence, we would like adjacent units of the network to organize
into similar features. This can be obtained by adding a smoothed $L^{1}$
penalty on the activation of the features over sub-regions of the
network. Hence, by summing over groups of features we ensure that
only localized groups activate for each example, leading to the overall
cost function

\[
\mathcal{J}(A,s)=\left\Vert As-x\right\Vert {}_{2}^{2}+\lambda\sum_{\text{all groups }g}\sqrt{\left(\sum_{\text{all }s\in g}s^{2}\right)+\epsilon}+\gamma\left\Vert A\right\Vert {}_{2}^{2}
\]


To easily compute this topographic penalty, we can rely on a \emph{grouping
matrix }$V$ which indicate for each row the membership of a feature
to a particular group. This allows to simplify the computation of
the gradients and provide an elegant way of writing the objective
function as
\[
J(A,s)=\left\Vert As-x\right\Vert {}_{2}^{2}+\lambda\sum_{r}\sum_{c}D_{r,c}+\gamma\left\Vert A\right\Vert {}_{2}^{2}
\]


where $D=\sqrt{Vss^{T}+\epsilon}$.


\subsection{Deconvolutional networks}

Deconvolutional networks are generative models which can be seen as
a convolutional form of sparse coding. One of the major difference
is that the pooling operation is differentiable and directly integrated
into the cost function via latent variables.


\subsubsection{Single layer}

The application of sparse coding yield an over-complete linear decomposition
of an input $\mathbf{x}$ using a set of basis vectors $\phi$ (with
a $L^{1}$ regularization to enforce sparsity). The convolutional
network is almost the exact opposite as this supervised and encoding-only
definition of sparse coding (by performing the steps of convolution,
non-linearity and pooling). De-convolution is based on trying to follow
the opposite path which seeks to find the feature maps (by trying
to find the unpooling variables) to reconstruct the input. Hence,
a de-convolutional network is a decoding trying to infer the features
by
\begin{enumerate}
\item Unpooling the feature maps (using inferred latent variables)
\item Convolving the unpooled maps (learned filters) to the input
\item Performing an unsupervised reconstruction of the input with sparsity
constraint
\end{enumerate}

\paragraph{Gaussian unpooling}

The principle behind de-convolutional networks is to try to \emph{unpool
}the observed features. Hence, each unpooling region can be seen as
a two-dimensional Gaussian with weights scaled by feature map activation.
The advantage of this formulation is that it leads to a differentiable
representation (with the unpooling variables being the mean and precisions
of the Gaussians, with one laid out for each feature map value).


\paragraph{Cost function}

The cost function for de-convolutional networks is closely similar
to that of sparse coding
\[
\frac{\lambda}{2}\left\Vert YU_{\theta}f-x\right\Vert _{2}^{2}+\left|f\right|_{\frac{1}{2}}
\]


with $x$ the input, $f$ the feature maps, $Y$ the reconstructed
input and $U_{\theta}$ is the unpooling operation (parametrized by
the means and precisions of the Gaussians).


\paragraph{Single layer inference}

The learning procedure of de-convolutional networks is also closely
related to the one used for sparse coding. Indeed, a two-step procedure
is used where one aspect is learned while the others are considered
fixed. The inference procedure tries to find
\begin{itemize}
\item Feature maps $p$ (the ``what'' aspect) are infered by fixing the
convolution filters and pooling variables $\theta$ (this amounts
to a convolutional form of sparse coding)
\item Pooling variables $\theta$ (the ``where'' aspect) are infered by
fixing the filters and feature maps, then running a chain rule of
derivatives to update the mean and precision of each Gaussian.
\end{itemize}

\subsubsection{Multi-layer stacking}

In order to produce a deep (multi-layer) deconvolutional network,
the idea is to use the pooled maps as input to next layer, but performing
a joint inference over all layers (which is only possible thanks to
the differentiable pooling operation). With the objective still being
to reconstruct the input image (to rely on the reconstruction error).
So the idea is still to minimize the reconstruction error of the input
image subject to sparsity. So multi-layer inference can be done by
updating feature maps at top but also pooling variables from \emph{both}
layers.


\section{Applying deep learning}


\subsection{Preprocessing}

The first step prior to trying to apply any learning mechanism to
a specific problem is to understand the data itself and carefully
try to understand its properties and inner structure. Indeed, datasets
of different nature will call for different pre-processing methods.
For instance, the \emph{stationnarity }property which is heavily used
in vision tasks does not necessarily hold for all types of data. Common
pre-processing methods for data normalization include 
\begin{itemize}
\item Rescaling the input data vectors can allow to bound the data in a
given range (usually $[0,1]$ or $[-1,1]$) to ensure the correct
use of different activation functions
\item Mean subtraction of examples is used when the data is \emph{stationary}
(the statistics of each dimension follow the same distribution). Subtracting
the mean-value of each training example can drive the algorithm towards
focusing on the variation of data rather than its absolute values
\item Feature standardization (through zero-mean, unit-variance transformation)
for each dimension across the dataset can allow to avoid that one
feature with a wider variation range overshadows the other components.
Hence, to balance the impact of each components, their values can
be rescaled independently.
\end{itemize}
Furthermore, dimensionality reduction techniques also provides the
advantage to significantly speed up the subsequent learning algorithm.
As a structured input is usually somewhat redundant (an amount of
correlation is always present in non-random data), can remove these
correlations to provide a lower dimensional version of the input,
while incurring only low amounts of error.


\subsubsection{PCA}

Principal Components Analysis (PCA) seeks to find the main axes of
variations of a set of vectors $\mathbf{x}$. Hence, this will require
to know their covariance matrix ${\textstyle \Sigma}$ which can be
computed (if $\mathbf{x}$ has a mean of zero) by 
\[
\Sigma=\frac{1}{n}\sum_{i=1}^{n}\mathbf{x}_{i}\mathbf{x}_{i}^{T}
\]
The principal directions of data variation are then defined by the
principal eigenvectors of ${\textstyle \Sigma}$. Based on this set
of vectors $U=[u_{1},\ldots u_{m}]$, we can rotate the original vector
$\mathbf{x}$ to obtain a set represented in this new basis

\[
x_{{\rm rot}}=U^{T}x
\]


As $U$ is an orthogonal matrix, we can go back and forth from this
representation in rotated space ($x_{{\rm rot}}$) to the original
space by computing $x=Ux_{{\rm rot}}$ (as $U$ satisfies $U^{T}U=UU^{T}=I$)

The dimensionnality reduction aspect implies that if we have $n$-dimensional
vectors $x\in\mathcal{R}^{n}$ and want to obtain $k$-dimensional
representation $\tilde{x}\in\mathbb{R}^{k}$ (where $k<n$), while
incuring a minimum amount of error, we could keep only the first $k$
components of $x_{{\rm rot}}$, which embeds the most important directions
of variation. In order to \uline{recover an approximation $\hat{x}$
of the original value of $x$ after this dimensionnality reduction,
we'd just multiply $\tilde{x}\in\mathbb{R}^{k}$ with the first $k$
columns of $U$.}

As different values of $k$ will imply different amounts of information
loss, we can analyze the percentage of variance retained by various
values of $k$. In order to automatically find the number of principal
components, we compute 
\[
\frac{\sum_{j=1}^{k}\lambda_{j}}{\sum_{j=1}^{n}\lambda_{j}}
\]


A common choice is to pick the $k$ principal components that retain
99\% of the variance.


\subsubsection{Stationarity}

PCA requires that each of the features lie in a similar range (variances
of different features are similar) and that their mean are close to
zero (usually obtained through zero-mean unit-variance transform).
However, estimating a separate mean and variance for each dimension
of the input vector seems quite senseless as the statistics in any
part of the input should be similar to that of any other part. This
property is called \emph{stationarity}.


\subsubsection{Whitening}

The preprocessing step known as \emph{whitening} (sometimes called
\emph{sphering}) is closely related to PCA and based on the same hypothesis
that the raw input has large redundancies stemming from highly correlated
neighborhoods. Similarly to PCA, whitening seeks to remove these redundancies
so that the learning algorithms are fed with training input in which
the features have low amounts of correlation and all exhibit the same
variance.

As the PCA already uncorrelates the features by providing a rotated
version of the input, we just need to ensure that each of the features
have unit variance. To do so, we can rescale each feature (principal
dimension) obtained with PCA by the scale of its variance ($1/\sqrt{\lambda_{i}}$).
Formally, we can obtain the whitened version of the data $x_{{\rm white}}\in\mathbb{R}^{n}$
as follows 
\[
x_{{\rm white}}^{i}=\frac{x_{{\rm rot}}^{i}}{\sqrt{\lambda_{i}}}
\]


The obtained $x_{white}$ is the whitened version of the input with
their different dimensions being uncorrelated and having unit variance.
As in some cases the eigenvalues $\lambda_{i}$ can be very close
to 0, the whitening step that divides rotated data by $\sqrt{\lambda_{i}}$
might lead to numerical errors. Therefore, this scaling step is usually
smoothed out by adding a small regularization constant $\epsilon$
to the eigenvalues before normalization 
\[
x_{{\rm white}}^{i}=\frac{x_{{\rm rot}}^{i}}{\sqrt{\lambda_{i}+\epsilon}}
\]



\subsubsection{ZCA Whitening}

It turns out that the transformation matrix that can be appled to
the data in order to obtain a unit covariance in all dimension (identity
matrix) is not unique. In fact, any orthogonal matrix $R$ (that satisfies
$RR^{T}=R^{T}R=I$) applied to the whitened data by $Rx_{{\rm white}}$
will lead to data with identity covariance. The ZCA whitening operation
is performed by choosing $R=U$ and applying 
\[
x_{{\rm ZCAwhite}}=Ux_{{\rm white}}
\]



\subsubsection{Pre-processing parameters}

When a variety of pre-processing operations are applied to training
data, it is mandatory to apply the same preprocessing parameters for
both the training and test phase. For instance, the rotation matrix
$U$ computing using PCA on the unlabeled training data should be
applied identically to preprocess the labeled test data. Indeed the
algorithm learns to exploit statistical regularities in this transformed
(rotated) space.


\subsection{Hyper-parameters analysis}

\textbf{NB NB : Should need a small ref per parameter : NB NB}


\subsubsection{The learning rate}

The learning rate controls the speed of learning. Although it would
seem natural to set this parameter to the maximal value, this increases
the risk of overshooting (where the update makes steps so wide that
it can ``jump over'' the optimal values). Oppositely, setting a
low learning rate will cause the algorithm to be very slow to converge.
An efficient way to trade between these two aspects is to gradually
decrease the learning rate over the training epochs (as the first
iterations have a higher chance of lying afar from a local optimum)
and average the weights update across various epochs to reduce the
learning noise.


\subsubsection{Initial weights values}

The initial values of the weights are crucial to the success of the
learning. First, it is mandatory not to initialize all weights to
zeros, as this will lead all error gradient to be identical across
the computation units and, therefore, the units will all end up. Hence,
any form of random initialization to small weights values also serve
the purpose of \emph{symmetry breaking}. Then, additionnal prior can
be embedded in the weight initialization. For instance, when a sparsity
penalty is imposed on the hidden units with a target probability of
$t$, it seems logical to initialize the hidden biases to $log\left(\nicefrac{t}{\left(1-t\right)}\right)$.


\subsubsection{Momentum}

Momentum has been proposed to speed up the learning procedure. The
main idea behind the momentum is to rely on a notion of velocity $v$
which is incremented at each step by the estimated gradient of the
parameters weighted by the learning rate and the previous velocity.
Hence, the momentum smooth the learning direction by taking into account
the gradient updates of previous mini-batches. Therefore, the velocity
is expected to decay with time given that a momentum meta-parameter
controls the influence of previous velocities in the current parameter
update 
\[
\Delta\theta_{i}\left(t\right)=v_{i}\left(t\right)=\alpha v_{i}\left(t-1\right)-\epsilon\frac{\partial E}{\partial\theta_{i}}\left(t\right)
\]


Using the method of momentum shifts the direction of the parameters
which is not that of steepest descent, making this method similar
to conjugate gradient approaches.


\subsubsection{Weight decay}

The weight decay penalty allows to avoid degeneracy caused by an unbounded
weight magnitude. This allows to impose a regularization by avoiding
equivalent weight updates and also prevents overfitting. However,
as the sum of all weights in a very large network might become extremely
strong (sum of millions of parameters), the weight decay coefficient
should be kept rather small (to ensure that it does not overshadow
the reconstruction error cost).


\subsubsection{Held-out validation data}

In order to select the appropriate amount of epochs and other factors
of success of the algorithm several stop criterion can be implemented.
These usually rely on held-out validation data, which allows to evaluate
the capacity of the trained network to generalize on novel data. 


\subsubsection{Encouraging sparse hidden activities}

It has been shown that discriminative performance can be improved
by using sparse features, i.e. only a very small portion of the units
are active for each given input vector \textbf{(REF: Nair and Hinton,
2009)}. The idea is that this sparsity would be the sign of the features
being sufficiently \emph{specific }(each object can be decomposed
in an efficiently small number of parts). One of the major differences
between auto-encoders and RBMs comes from this sparsity requirement.
Indeed, RBMs should not require the addition of a sparsity regularization
term to encourage sparse activities because their use of stochastic
binary hidden units already acts as a very strong regularizer. However,
it is possible to introduce an extra sparsity constraint to further
control the learning objective of RBMs \cite{lee2008sparse}.

This objective by setting a desired \emph{sparsity target }which indicates
the desired percentage of activation of a unit over the complete training
set. The sparsity penalty term is therefore calculated as the difference
between the actual activation probability and this desired target.
This can be estimated with $q$ being the mean activation probability
and performing an exponentially decaying overage over the different
mini-batches 
\[
q_{new}=\lambda q_{old}+(1\text{\textminus}\lambda)q_{current}
\]


The natural penalty measure to use is the cross entropy between the
desired sparsity $t$ and actual distribution $q$ 
\[
S_{penalty}\propto-t.log\left(q\right)-\left(1-t\right)log\left(1-q\right)
\]


This penalty shows the advantage of having a simple derivative for
each unit. Finally, this penalty is usually scaled by a \emph{sparsity
cost }parameter which set the impact of the sparsity on the overall
cost. In order to efficiently set this parameter, histograms of the
activities and careful initialization of the weights can allow to
reduce the initial impact of this sparsity in order to increase its
cost, without having it interfere with the reconstruction objective.


\subsubsection{Number of hidden units}

Recently, a growing consensus in the deep learning community seems
to point out that the architecture of the networks may play the most
important part in the success of the learning. It has even been shown
that networks with random weights had highly correlated accuracy with
trained networks with similar architecture on classification problems
\cite{saxe2011random}. Hence, it appears that the network architecture
and connection pattern might have a more crucial importance than parameter
initialization.

\textbf{+ Turns this paragraph into COMPLEXITY / Width vs. Depth /
E.L.M / Full essay on structure}

\uline{When learning generative models of high-dimensional data,
however, it is the number of bits that it takes to specify a data
vector that determines how much constraint each training case imposes
on the parameters of the model. This can be several orders of magnitude
greater than number of bits required to specify a label. This would
allow 1000 globally connected hidden units. If the hidden units are
locally connected or if they use weight-sharing, many more can be
used.}


\subsubsection{Varieties of contrastive divergence}

It has been shown that using a single step of contrastive divergence
(CD1) provides surprisingly good performance \cite{hinton2006fast}.
However, it might also makes sense to increase the number of Gibbs
sampling up to $n$ steps (CDn) \cite{carreira2005contrastive}. It
has been shown that a practical way to balance this number of steps
is to gradually increase this $n$ as the weights grow \cite{salakhutdinov2007restricted}.
Another even more different way of learning is called \emph{persistent
contrastive divergence }\cite{tieleman2008training}, which use persistent
chains (called \emph{fantasy particles}) to initialize the Gibbs Markov
chain instead of resetting it to a datavector at ecah iteration. The
learning is then done by computing the difference in pairwise statistics
on a minibatch and the pairwise statistics on these persistent chains.
This approach has been improved by adding a set of ``fast weights''
overlaid on the standard parameters \cite{tieleman2009using}.


\subsubsection{The size of a mini-batch}

A common efficient way to speed up the learning process is to divide
the complete training set into small \textquotedblleft \emph{mini-batches}\textquotedblright{}
(a random small subset of the entire input dataset) and perform one
gradient step per batch. This allows both to perform a lot larger
number of gradient updates (which improves the exploration of the
solution space), and also to reduce the complexity of a single step
(allowing tractable matrix operations to run on GPU boards). However,
this would imply that different sizes of minibatch could impact the
grandient descent optimization. Hence, to avoid changing the learning
rate concomitently with the size of a mini-batch, the gradient computed
on a mini-batch is averaged by its size.

The optimal size of a mini-batch is often given by the number of classes.
To reduce the sampling error when estimating the gradient for the
whole training set, each mini-batch should contain at least one instance
from each class \cite{hinton2010practical}.


\subsection{Parameter tuning search}


\subsection{Monitoring and displaying}


\subsubsection{Monitoring the learning}

Monitoring the learning is usually done by simply plotting the evolution
of the reconstruction error. Given the nature of the learning objective,
it would seem logical to monitor the error between the data and the
reconstructions performed by the network during learning. However,
it turns out that the reconstruction error is a quite poor indicator
of the learning progress. For AEs, the reconstruction error does not
faithfully reflects the additionnal constraints such as weight decay
or sparsity. Therefore, this error provides no information on the
\emph{quality }of the learning (for instance, the identity function
would give a perfect reconstruction but a useless learning). For RBMs,
the reconstruction error is not at all the function that CDn is trying
to approximate. Hence, it confounds together the difference between
distributions of data and equilibrium distribution and the mixing
rate. Therefore, alternate types of display can give lot deeper insights
on the behavior and evolution of learning in deep networks rather
than simply monitoring the value of the reconstruction error. 


\paragraph{Histograms}

By plotting histograms of the weights and biases, it is possible to
quickly detect a degeneracy in the learning. The same type of scrutinization
can be done by plotting the histogram of increments in the parameters,
as this shows if the learning perform strong updates of the parameters,
and even if updates are done at all.


\paragraph{Specificity of the features}

As we hope to discover specific features, in which an input is efficiently
decomposed in a sparse number of hidden features (which is the point
of sparsity penalties), we can monitor the activation patterns over
the datasets. Hence, a two-dimensionnal display of the activation
probabilities where each row represent a hidden unit and each column
an example allows to directly see if some units are unused or convertly
activated over a large part of the datasets (the same apply to examples).
This can show both the \emph{specificity }of the units but also their
\emph{certainty }towards the feature they learned.


\paragraph{Weights display}

An interesting way to understand the inner behavior of the network
and more importantly what features have been learned by the hidden
units is to display their weights. Knowing the weights $W_{ij}$ learned
by a hidden unit, we can compute the input which maximally activates
this unit by 
\[
x_{j}=\frac{W_{ij}}{\sqrt{\left\Vert W_{i}\right\Vert ^{2}}}
\]
The normalizing term comes from the fact that the input is constrained
by the $L^{2}$ norm. By displaying this information with the same
procedure that would be used for the raw input data, we can understand
which feature each of the hidden unit is looking for. Indeed, these
show what maximally activates the unit so in turn shows what feature
has been learned through expressing the correlations they are seeking.
Therefore, these type of display can be interpreted as a form of \emph{receptive
field}.


\subsubsection{t-distributed Stochastic Neighbor Embedding (t-SNE)}

\textbf{SUMMARIZE THE FIRST ARTICLE on t-SNE.}


\section{Applications}

Deep learning approaches have been appliued to a whole range of machine
learning problems \textbf{{[}xREFx{]}}, mainly in computer vision
\textbf{{[}xREFx{]} }but also in more disparate fields ranging from
audio processing \cite{yu2011deep} to natural language processing
\textbf{{[}xREFx{]} }and even automatic game playing \cite{mnih2015human}.
Even though applications have been flourishing in the conventional
analysis, recognition and classification fields, the scope of deep
processing have been gradually extended \uline{to include more
human-centric tasks of interpretation, understanding, retrieval, mining,
and user interface }\textbf{\uline{{[}xREFx{]}}}

Stacked auto-encoders have been used for speech compression problem
with the aim of fitting the data to a fixed number of bits while minimizing
the reproduction error \cite{deng2010binary}. The pretraining step
of the DBN is shown to be crucial for increasing the coding efficiency.

\uline{In \mbox{\cite{nair20093d}}, Nair and Hinton developed
a modified DBN where the top-layer model uses a third-order Boltzmann
machine. }\textbf{\uline{+ FOR WHAT (or remove)}}

Another way to rely on DBNs and deep autoencoder is to consider their
ability to transform the input data to a higher-level representation.
This approach was investigated for document indexing and retrieval
{[}\textbf{xREFx}{]}, {[}\textbf{xREFx{]}}, where ut is shown that
the deepest hidden variables transforming the input based on word
count features provides strongly better representations of each document.
Furthermore, similarity in this hidden representation mirrors the
semantic similarity of text documents, which facilitates rapid document
retrieval.


\section{Future directions}

\cite{yu2011deep}:

\uline{We need to better understand the deep model and deep learning.
Why is learning in deep models difficult? Why do the generative pretraining
approaches seem to be effective empirically? Is it possible to change
the underlining probabilistic models to make the training easier?
Are there other more effective and theoretically sound approaches
to learn deep models?}

\uline{We need to find better feature extraction models at each
layer}

\uline{This suggests that the current Gaussian-Bernoulli layer
is not powerful enough to extract important discriminative information
from the features.}

\uline{Theory needs to be developed to guide the search of proper
feature extraction models at each layer.}

\uline{It is necessary to develop more powerful discriminative
optimization techniques.}

\uline{The features extracted at the generative pretraining phase
seem to describe the underlining speech variations well but do not
contain enough information to distinguish between different languages.
A learning strategy that can extract discriminative features for those
tasks is in need. Extracting discriminative features may also greatly
reduce the model size}

\uline{We need to develop effective and scalable parallel algorithms
to train deep models. The current optimization algorithm, which is
based on the mini-batch stochastic gradient, is difficult to be parallelized
over computers. To make deep learning techniques scalable to thousands
of hours of speech data, for example, theoretically sound parallel
learning algorithms need to be developed.}

\uline{We need to search for better approaches to use deep architectures
for modeling sequential data. The existing approaches, such as DBNHMM
and DBN-CRF, represent simplistic and poor temporal models in exploiting
the power of DBNs. Models that can use DBNs in a more tightly integrated
way and learning procedures that optimize the sequential criterion}

\uline{Developing adaptation techniques for deep models is necessary.
Many conventional models such as GMMHMM have well-developed adaptation
techniques that allow for these models to perform well under diverse
and changing real-world environments}

\uline{\mbox{\cite{humphrey2013feature}}:}

\uline{some research in feature learning has shown it is possible
to discover unexpected attributes that are useful to well-worn problems}

\uline{An unexpected consequence of this work is the realization
that certain feature detectors, learned from constant-Q representations,
seem to encode distinct pitch intervals.}

\uline{the long-standing assumption in feature design for genre
recognition is that spectral contour matters much more than harmonic
information}

\uline{e significantly improved upon by adding a musically motivated
sequence model after the pattern recognition stage to smooth classification}

\uline{the work presented in \mbox{\cite{humphrey2012rethinking}}
adopts a slightly different view of chord recognition. Using a CNN
to classify five-second tiles of constant-Q pitch spectra, an end-to-end
chord recognition system is produced that considers context from input
observation to output label. Figure 10 illustrates how receptive fields,
or local feature extractors, of a convolutional network build abstract
representations as the hierarchical composition of parts over time.}

\uline{A hierarchy of harmony: Operating on five-second CQT patches
as an input (i), the receptive fields of a CNN encode local behavior
in feature maps (ii) at higher levels. This process can then be repeated
(iii), allowing the network to characterize high-level attributes
as the combination of simpler parts. This abstract representation
can then be transformed into a probability surface (iv) for classifying
the input}

\uline{First and foremost, building context into the feature representation
greatly reduces the need for post-filtering after classification.
Therefore, this accuracy is achieved by a causal chord recognition
system}

\uline{updating this perspective is through discussions like this
one, by demystifying the proverbial \textquotedblleft black box\textquotedblright{}
and understanding what, how, and why these methods work. Additionally,
reframing traditional problems in the viewpoint of deep learning serves
as an established starting point to begin developing a good comprehension
of implementing and realizing these systems.}

\uline{formulate novel or alternative theoretical foundations.}

\uline{successful application of deep learning necessitates a thorough
understanding of these methods and how to apply them to the problem
at hand. Various design decisions, such as model selection, data pre-processing,
and carefully choosing the building blocks of the system, can impact
performance on a continuum from negligible differences in overall
results to whether or not training can, or will, converge to anything
useful. Likewise, the same kind of intuition holds for adjusting the
various hyperparameters \textemdash learning rate, regularizers, sparsity
penalties}

\uline{the approach is overtly more abstract and conceptual, placing
a greater emphasis on high-level decisions like the choice of network
topology or appropriate loss function}

\uline{, it is prudent to recognize that the majority of progress
has occurred in computer vision. While this gives our community an
excellent starting point, there are many assumptions inherent to image
processing that start to break down when working with audio signals.}

\uline{the strongest correlations in an image occur within local
neighborhoods, and this knowledge is reflected in the architectural
design. Local neighborhoods in frequency do not share the same relationship,
so the natural question becomes, \textquotedblleft what architectures
do make sense for time-frequency representations?\textquotedblright{}}

\uline{In turn, such approaches may subsequently provide insight
into the latent features that inform musical judgements, or even lead
to deployable systems that could adapt to the nuances of an individual.}

\uline{the datadriven prior that it leverages can be steered by
creating specific distributions, e.g., learn separate priors for rock
versus jazz. Finally, music signals provide an interesting setting
in which to further explore the role of time in perceptual AI systems,
and has the potential to influence other time-series domains like
video or motion capture data.}

\uline{\mbox{\cite{rifai2011contractive}}:}

\uline{much remains to be done to understand the characteristics
and theoretical advantages of the representations learned by a Restricted
Boltzmann Machine \mbox{\cite{hinton2006fast}}, an auto-encoder \mbox{\cite{bengio2007greedy}},
sparse coding \mbox{\cite{olshausen1997sparse,kavukcuoglu2009learning}},
or semi-supervised embedding }


\section{General infos}

DARPA deep learning program, available at http://www.darpa. mil/ipto/solicit/baa/BAA-09-40\_PIP.pdf).

\bibliographystyle{plain}
\bibliography{deeplearning}

\end{document}
