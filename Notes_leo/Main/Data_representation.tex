\documentclass{report}
\usepackage{framed}
\usepackage[outerbars]{changebar}

% Dimensions de la page
\usepackage{geometry}
%\geometry{scale=0.8}
\geometry{a4paper}

% Acronyms
\usepackage[nonumberlist]{glossaries}
\setacronymstyle{long-short}
\makenoidxglossaries
% Load acronyms list
\loadglsentries{acronyms}

% Graphics
\usepackage{graphicx}
\graphicspath{{Figures/}}

% Dimensions de la page
\usepackage{geometry}
%\geometry{scale=0.8}
\geometry{a4paper}

% Acronyms
\usepackage[nonumberlist]{glossaries}
\setacronymstyle{long-short}
\makenoidxglossaries
% Load acronyms list
\loadglsentries{acronyms}

% Graphics
\usepackage{graphicx}
\graphicspath{{Figures/}}

% Prettyref
\usepackage{prettyref}
\usepackage{hyperref}

\title{Data representation}
\author{Aciditeam}
\begin{document}
\maketitle

\section{Pianoroll representation}
\subsection{Binary pianoroll}
The first and most common data representation we can think about is a matrix of binary units indicating whether a note is on or off. In this case, the network would use binary units.

\subsection{Dynamic pianoroll}
The first improvement we can make is to take dynamics into consideration. I think that dynamics can provide a useful information about the position of beat/down-beats and be used to infer a rhythmic structure.

This imply using a network with real valued input units, either by considering probabilities as the final value, or even using Gaussian or preferably ReLu units.

\section{Braids}
Using the braids representation (cf Mattia).
Still to be read, be very promising in order to obtain a compact meaningful representation. A priori rely on real units.

\section{From sparse binary encoding to compact distributional representations}



\end{document}
